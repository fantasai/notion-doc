
\chapter{Graphical styles}
\label{chap:gr}

This chapter first gives in section \ref{sec:engines} a general outline 
of how drawing engines are used, of style specifications and then
in section \ref{sec:defaultde} describes how to specify styles
for the default drawing engine. Some additional settings and
user attributes are explained in Sections \ref{sec:grmisc}.


\section{Drawing engines, style specifications and sub-styles}
\label{sec:engines}
\index{style}\index{drawing engine}

Ion's drawing routines are abstracted into so-called drawing engine
modules that can, again depending on the system, be dynamically
loaded as needed. The drawing engine modules provide ``brushes''
that objects can use to draw some high-level primitives such
as borders and text boxes (in addition to simple text and rectangle
drawing) on their windows and configure e.g. the shape and 
background of the window. While the drawing engines therefore
do not directly implement looks for each possible object (that
would hardly be maintainable), different brush styles can be
used to give a distinctive look to different objects and engines
could interpret some styles as special cases. Style specifications 
are strings of the form

\begin{verbatim}
element1-element2-...-elementn
\end{verbatim}

An example of such a style specification is \codestr{tab-frame};
see the table in subsection \ref{sec:styles} for more styles.

When an object asks for a brush of certain style, the selected
drawing engine will attempt to find the closest match to this
specification. The styles/brushes defined by the drawing engines 
may have asterisks (\codestr{*}) as some of the elements indicating
a match to anything. Exact matches are preferred to asterisk
matches and longer matches to shorter. For example, let a brush
for style \codestr{foo-bar-baz} be queried, then the following
brushes are in order of preference:

\begin{verbatim}
foo-bar-baz
foo-*-baz
foo-bar
*
\end{verbatim}

Some of the drawing primitives allow extra attributes to be
specified, also in the form
\begin{verbatim}
attr1-attr2-...-attrn
\end{verbatim}
These extra attributes are called \emph{substyles}\index{substyle}
and allow, for example, the state of the object to be indicated 
by  different colour sets while keeping the interface at an 
abstract level and the drawing engine completely ignorant 
of the semantics -- only the writer of the drawing engine
configuration file has to know them. However the drawing 
engine can again interpret known substyles as special cases
and the default engine indeed does so with frame tab 
tag and drag states.)


\subsection{Known styles and substyles}
\label{sec:styles}

\subsubsection{Frames}

\begin{tabularx}{\linewidth}{lX}
\tabhead{Style name & Description}
\codestr{frame} & Style for frames. 
	Substyle attributes: \codestr{active}/\codestr{inactive} 
	(mutually exclusive) and
	\codestr{quasiactive}/\codestr{not\_quasiactive}. 
	A frame is ``quasiactive'' when an active region
	has a back-link to it, such as a detached window. \\
\codestr{frame-tiled} & A more specific style for tiled frames.
        Substyle attributes as for \codestr{frame}. \\
\codestr{frame-tiled-alt} & An alternative style for tiled frames.
        Often used to disable the tab-bar. \\
\codestr{frame-floating} & A more specific style for floating
        frames. \\
\codestr{frame-transient} & A more specific style for frames
        containing transient windows. \\
\end{tabularx}

\subsubsection{Tabs and menu entries}

\begin{tabularx}{\linewidth}{lX}
\tabhead{Style name & Description}
\codestr{tab} & Style for frames' tabs and menu entries. 
	Substyle attributes:
	\codestr{active}/\codestr{inactive} and
	\codestr{selected}/\codestr{unselected} \\
\codestr{tab-frame} & A more specific style for frames' tabs.
        Additional substyle attributes include:
      	\codestr{tagged}/\codestr{not\_tagged},
	\codestr{dragged}/\codestr{not\_dragged},
	\codestr{activity}/\codestr{no\_activity},
	\codestr{quasiactive}/\codestr{not\_quasiactive}. \\
\codestr{tab-frame-tiled}, & \\
\codestr{tab-frame-tiled-alt}, & \\
\codestr{tab-frame-floating}, & \\
\codestr{tab-frame-transient} & More specific styles for frames in the
        different modes. \\
\codestr{tab-menuentry} & A more specific style for entries in \type{WMenu}s. 
        Additional substyle attributes include \codestr{submenu} and
        occasionally also \codestr{activity} is used.\\
\codestr{tab-menuentry-bigmenu} & 
        An alternate style for entries in \type{WMenu}s. \\
\end{tabularx}

\subsubsection{The rest}

\begin{tabularx}{\linewidth}{lX}
\tabhead{Style name & Description}
\codestr{input} & A style for \type{WInput}s. \\
\codestr{input-edln} & A more specific style for \type{WEdln}s. 
	Substyle attributes: \codestr{selection} for selected text and
	\codestr{cursor} for the cursor indicating current editing point. \\
\codestr{input-message} & A more specific style for \type{WMessage}s. \\
\codestr{input-menu} & A more specific style for \type{WMenu}s. \\
\codestr{input-menu-bigmenu} & An alternate style for \type{WMenu}s. \\
\codestr{moveres\_display} & The box displaying position/size when
	moving or resizing frames. \\
\codestr{stdisp} & Any status display. \\
\codestr{stdisp-dock} & The dock. \\
\codestr{stdisp-statusbar} & The statusbar. Substyles include:
    the name of any monitor/meter (such as \codestr{date}), and 
    the supplied hint. Typical hints are: \codestr{important} 
    and \codestr{critical}. \\
\end{tabularx}


\section{Defining styles for the default drawing engine}
\label{sec:defaultde}

Drawing engine style files are usually named
\file{look\_foo.lua} where \file{foo} is the name of the
style. The file that Ion loads on startup or when
\fnref{gr.read_config} is called, however, is \file{look.lua}
and should usually be symlinked to or a copy of of some
\file{look\_foo.lua}.

\subsection{The structure of the configuration files}

The first thing to do in a style file is to choose the drawing
engine, possibly loading the module as well. This is done
with the following chunk of code.

\begin{verbatim}
if not gr.select_engine("de") then 
    return 
end
\end{verbatim}

The \fnref{gr.select_engine} function sees if the engine
given as argument is registered (the default drawing engine is
simply called ``de''). If the engine could not be found, it
tries to load a module of the same name. If the engine still
is not registered, \fnref{gr.select_engine} returns \codestr{false}
and in this case we also exit the style setup script.
If the engine was found, \fnref{gr.select_engine} sees that
further requests for brushes are forwarded to that engine
and returns \codestr{true}.

Before defining new styles it may be a good idea to clear old
styles from memory so if the old configuration defines more
specific styles than the new, the old styles don't override 
those specified by the new configuration. That can be done by
calling

\begin{verbatim}
de.reset()
\end{verbatim}

After this the new styles can be defined with \fnref{de.defstyle}
as explained in the next subsection. Finally, after the styles have
been defined we must ask objects on the screen to look up new brushes
to reflect the changes in configuration. This is done with

\begin{verbatim}
gr.refresh()
\end{verbatim}

\subsection{Defining the styles}

Styles for the default drawing engine are defined with the
function \fnref{de.defstyle}. It has two arguments the first being
a style specification as explained in previous sections and the second
a table whose fields describe the style:

\begin{verbatim}
de.defstyle("some-style", {
    attribute = value,
    ...
})
\end{verbatim}

The supported attributes are described in tables below. The different
border elements and styles referred to there are explained in Figure
\ref{fig:borders}.

\begin{figure}
\begin{htmlonly}
\docode % Kludge to make latex2html interpret contents instead of
	% creating an image.
\end{htmlonly}    
\begin{verbatim}
Elevated:        Inlaid:         Ridge:            Groove:        
  hhhhhhhhhhhs     ............	   hhhhhhhhhhhs      sssssssssssh
  h..........s     .sssssssssh.	   h..........s      s..........h
  h.        .s     .s        h.	   h.sssssssh.s      s.hhhhhhhs.h
  h.        .s     .s        h.	   h.s      h.s      s.h      s.h
  h.        .s     .s        h.	   h.shhhhhhh.s      s.hsssssss.h
  h..........s     .shhhhhhhhh.	   h..........s      s..........h
  hsssssssssss     ............	   hsssssssssss      shhhhhhhhhhh

h = highlight, s = shadow, . = padding
\end{verbatim}
\caption{Sketch of different border styles and elements}
\label{fig:borders}
\end{figure}

\subsubsection{Colours}

Each of these fields a string of the form that can be
passed to \code{XAllocNamedColor}. Valid strings are e.g.
hexadecimal RGB specifications of the form
\code{#RRGGBB} and colour names as specified
in \file{/usr/X11R6/lib/X11/rgb.txt} (exact path varying).

\begin{tabularx}{\linewidth}{lX}
\tabhead{Field  & Description}
\var{highlight_colour} &  
	Colour for the ``highlight'' part of a border. \\
\var{shadow_colour}    &  
	Colour for the ``shadow'' part of a border. \\
\var{foreground_colour} &  
	Colour for the normal drawing operations, e.g. text. \\
\var{background_colour} &  
	Window background colour (unless transparency is enabled) and
	background colour boxes. \\
\var{padding_colour} &  
	Colour for the ``padding'' part of a border border. Set to 
	\var{background_colour} if unset. \\
\end{tabularx}


\subsubsection{Borders and widths}

All other fields below except \var{border_style} are non-negative integers
indicating a number of pixels.

\begin{tabularx}{\linewidth}{lX}
\tabhead{Field  & Description}
\var{border_style} & A string indicating the style of border; one of
		     \codestr{elevated}/\codestr{inlaid}/\codestr{ridge}/\codestr{groove}
		     as seen in the above sketch. \\
\var{border_sides} & A string indicating which sides of the border
                     to draw: \codestr{all}/\codestr{tb}/\codestr{lr} for all, 
                     top and bottom, and left and right. To control between
                     left/right and top/bottom, use the pixel options below. \\
\var{highlight_pixels} &  
	Width of the highlight part of the border in pixels. \\
\var{shadow_pixels}    &  
	Width of the shadow part of the border in pixels. \\
\var{padding_pixels} &  
	Width of the padding part of the border in pixels. \\
\var{spacing} &
	Space to be left between all kinds of boxes. \\
\end{tabularx}


\subsubsection{Text}

\begin{tabularx}{\linewidth}{lX}
\tabhead{Field  & Description}
\var{font} & Font to be used in text-drawing operations; standard X font
	     name. \\
\var{text_align} & How text is to be aligned in text boxes/tabs; one of
		   the strings \codestr{left}/\codestr{right}/\codestr{center}. \\
\end{tabularx}


\subsubsection{Miscellaneous}


\begin{tabularx}{\linewidth}{lX}
\tabhead{Field  & Description}
\var{transparent_background} & Should windows' that use this style
	background be transparent? true/false. \\
\var{based_on} & The name of a previously defined style that this
	style should be based on. \\
\end{tabularx}


\subsubsection{Substyles}

As discussed in previous sections, styles may have substyles to e.g.
indicate different states of the object being drawn. The ``de'' engine
limits what can be configured in substyles to the set of colours in the
first table above, but also specifically interprets for the main style
\codestr{tab-frame} the substyles \codestr{*-*-tagged} and \codestr{*-*-*-dragged}
by, respectively, drawing a right angle shape at the top right corner 
of a tab and by shading the tab with a stipple pattern. Also for
menus the substyles \codestr{*-*-submenu} are handled as a special case.

Substyles are defined with the function \fnref{de.substyle} within the
table defining the main style. The parameters to this function are 
similar to those of \fnref{de.defstyle}.

\begin{verbatim}
de.defstyle("some-style", {
   ...
   de.substyle("some-substyle", {
      ...
   }),
   ...
})
\end{verbatim}


\subsection{An example}

The following shortened segment from \file{look\_cleanviolet.lua}
should help to clarify the matters discussed in the previous
subsection.

\begin{verbatim}
de.defstyle("*", {
    -- Gray background
    highlight_colour = "#eeeeee",
    shadow_colour = "#eeeeee",
    background_colour = "#aaaaaa",
    foreground_colour = "#000000",
    
    shadow_pixels = 1,
    highlight_pixels = 1,
    padding_pixels = 1,
    spacing = 0,
    border_style = "elevated",
    
    font = "-*-helvetica-medium-r-normal-*-12-*-*-*-*-*-*-*",
    text_align = "center",
})

de.defstyle("tab-frame", {
    based_on = "*",
    
    de.substyle("active-selected", {
        -- Violet tab
        highlight_colour = "#aaaacc",
        shadow_colour = "#aaaacc",
        background_colour = "#666699",
        foreground_colour = "#eeeeee",
    }),

    --  More substyles would follow ...
})
\end{verbatim}


\section{Miscellaneous settings}
\label{sec:grmisc}


\subsection{Frame user attributes}

The function \fnref{WFrame.set_grattr} may be used to give frames
(and their tabs) arbitrary extra attributes to be passed to the
drawing engine. Hence, by configuring such substyles in the style
configuration files, and turning on the attribute when needed, 
scripts may display visual cues related to the frame. There is
also one extra attribute specially interpreted by the default
drawing engine: the \codestr{numbered} attribute, which causes 
numbers to be displayed on the tabs.


\subsection{Extra fields for style \codestr{frame}}

The following style fields are independent of the drawing engine used,
but are related to objects' styles and therefore configured in the drawing
engine configuration file.

\begin{tabularx}{\linewidth}{lX}
\tabhead{Field  & Description}
\code{bar} & Controls the style of the tab-bar. Possible values
        are the strings \codestr{none}, \codestr{inside}, \codestr{outside}
        and \codestr{shaped}, with the last providing the PWM-style
        tab-bars for floating frames. \\
\code{floatframe_tab_min_w} & Minimum tab width in pixels for
        the shaped style, given that this number times number of tabs
        doesn't exceed frame width. \\
\code{floatframe_bar_max_w_q} & Maximum tab-bar width quotient of
	frame width for the shaped styles. A number in the 
	interval $(0, 1]$.
\end{tabularx}



\subsection{Extra fields for style \codestr{dock}}

\begin{tabularx}{\linewidth}{lX}
\tabhead{Field  & Description}
\code{outline_style} & How borders are drawn:
			\codestr{none} -- no border,
			\codestr{all} -- border around whole dock,
			\codestr{each} -- border around each dockapp. \\
\code{tile_size} & A table with entries \codestr{width} and \codestr{height},
	indicating the width and height of tiles in pixels.
\end{tabularx}


Hopefully that's enough to get you started in writing new style
configuration files for Ion. When in doubt, study the existing
style configuration files.
