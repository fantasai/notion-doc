
\chapter{Graphical styles}

This chapter first gives in section \ref{sec:engines} a general outline 
of how drawing engines are used, of style specifications and then
in section \ref{sec:defaultde} describes how to specify styles
for the default drawing engine.

\section{Drawing engines, style specifications and substyles}
\label{sec:engines}
\index{style}\index{drawing engine}

Ion's drawing routines are abstracted into so-called drawing engine
modules that can, again depending on the system, be dynamically
loaded as needed. The drawing engine modules provide ''brushes''
that objects can use to draw some high-level primitives such
as borders and text boxes (in addition to simple text and rectangle
drawing) on their windows and configure e.g. the shape and 
background of the window. While the drawing engines therefore
do not directly implement looks for each possible object (that
would hardly be maintainable), different brush styles can be
used to give a distinctive look to different objects and engines
could interpret some styles as special cases. Style specifications 
are strings of the form

\begin{verbatim}
element1-element2-...-elementn
\end{verbatim}

An example of such a style specification is \code{frame-tab};
see the table in subsection \ref{sec:styles} for more styles.

When an object asks for a brush of certain style, the selected
drawing engine will attempt to find the closest match to this
specification. The styles/brushes defined by the drawing engines 
may have asterixes (\verb!*!) as some of the elements indicating
a match to anything. Exact matches are preferred to asterisk
matches and longer matches to shorter. For example, let a brush
for style \code{foo-bar-baz} be queried, then the following
brushes are in order of precedence:

\begin{verbatim}
foo-bar-baz
foo-*-baz
foo-bar
*
foo-baz   -- Doesn't match, not selected!
\end{verbatim}

Some of the drawing primitives also extra attributes to be
specified also in the form
\begin{verbatim}
attr1-attr2-...-attrn
\end{verbatim}
These extra attributes are called \emph{substyles}\index{substyle}
and allow, for example, the state of the object to be indicated 
by  different colour sets while keeping the interface at an 
abstract level and the drawing engine completely ignorant 
of the semantics -- only the writer of the drawing engine
configuration file has to know them. However the drawing 
engine can again interpret known substyles as special cases
and the default engine indeed does so with frame tab 
tag and drag states.)


\subsection{Known styles and substyles}
\label{sec:styles}

\begin{tabularx}{\linewidth}{lX}
\hline
Style name & Description \\
\hline
\code{frame} & Style for frames. 
	Substyles: \code{active}, \code{inactive}. \\
\code{frame-ionframe} & A more specific style for \type{WIonFrame}s. 
	Substyles as for \code{frame}. \\
\code{frame-floatframe} & A more specific style for \type{WFloatFrame}s. 
	Substyles as for \code{frame}. \\
\code{frame-tab} & Style for frames' tabs. 
	Substyles: combinations of the form \code{a-s-t-d-u} where
	\code{a} is one of \code{active}/\code{inactive},
	\code{s} is one of \code{selected}/\code{unselected},
      	\code{t} is one of \code{tagged}/\code{not_tagged},
	\code{d} is one of \code{dragged}/\code{not_dragged} and 
	\code{u} is one of \code{urgent}/\code{not_urgent}. \\
\code{frame-tab-ionframe} & A more specific style for \type{WIonFrame}s' 
	tabs. Substyles as for \code{frame-tab}. \\
\code{frame-tab-floatframe} & A more specific style for 
	\type{WFloatFrame}s' tabs. Substyles as for \code{frame-tab}. \\
\code{input} & A style for \type{WInput}s. \\
\code{input-edln} & A more specific style for \type{WEdln}s. 
	Substyles: \code{selection} for selected text and
	\code{cursor} for the cursor indicating current editing point. \\
\code{input-message} & A more specific style for \type{WMessage}s. \\
\code{moveres_display} & The box displaying position/size when
	moving or resizing frames. \\
\end{tabularx}


\section{Defining styles for the default drawing engine}
\label{sec:defaultde}

Drawing engine style files are usually named
\file{look-foo.lua} where \file{foo} is the name of the
style. The file that Ion loads on startup or when
\fnref{gr_read_config} is called, however, is \file{draw.lua}
and should usually be symlinked to or a copy of of some
\file{look-foo.lua}.

\subsection{The structure of the configuration files}

The first thing to do in a stylefile is to choose the drawing
engine, possibly loading the module as well. This is done
with the following chunk of code.

\begin{verbatim}
if not gr_select_engine("de") then 
    return 
end
\end{verbatim}

The \fnref{gr_select_engine} function sees if the engine
given as argument is register (the default drawing engine is
simply called ''de''). If the engine could not be found, it
tries to load a module of the same name. If the engine still
is not registered, \fnref{gr_select_engine} returns \code{false}
and in this case we also exit the script style setup script.
If the engine was found, \fnref{gr_select_engine} sees that
further requests for brushes are forwarded to that engine
and returns \code{true}.

Before defining new styles it may be a good idea to clear old
styles from memory so if the old configuration defines more
specific styles than the new, the old styles don't override 
those specified by the new configuration. That can be done by
calling

\begin{verbatim}
de_reset()
\end{verbatim}

After this the new styles can be defined with \fnref{de_define_style}
as explained in the next subsection. Finally, after the styles have
been defined we must ask objects on the screen to look up new brushes
to reflect the changes in configuration. This is done with

\begin{verbatim}
gr_refresh()
\end{verbatim}

\subsection{Defining the styles}

Styles for the default drawing engine are defined with the
function \fnref{de_define_style}. It has two arguments the first being
a style specification as explained in previous sections and the second
a table whose fields describe the style:

\begin{verbatim}
de_define_style("some-style", {
    attribute = value,
    ...
})
\end{verbatim}

The supported attributes are described in tables below. The different
border elements and styles referred to there are explained in Figure
\ref{fig:borders}.

\begin{figure}
\begin{htmlonly}
\docode % Kludge to make latex2html interpret contents instead of
	% creating an image.
\end{htmlonly}    
\begin{verbatim}
Elevated:        Inlaid:         Ridge:            Groove:        
  hhhhhhhhhhhs     ............	   hhhhhhhhhhhs      sssssssssssh
  h..........s     .sssssssssh.	   h..........s      s..........h
  h.        .s     .s        h.	   h.sssssssh.s      s.hhhhhhhs.h
  h.        .s     .s        h.	   h.s      h.s      s.h      s.h
  h.        .s     .s        h.	   h.shhhhhhh.s      s.hsssssss.h
  h..........s     .shhhhhhhhh.	   h..........s      s..........h
  hsssssssssss     ............	   hsssssssssss      shhhhhhhhhhh

h = highlight, s = shadow, . = padding
\end{verbatim}
\caption{Sketch of different border styles and elements}
\label{fig:borders}
\end{figure}

\subsubsection{Colours}

Each of these fields a string of the form that can be
passed to \code{XAllocNamedColor}. Valid strings are e.g.
hexadecimal RGB specifications of the form
\code{#RRGGBB} and colour names as specified
in \file{/usr/X11R6/lib/X11/rgb.txt} (exact path varying).

\begin{tabularx}{\linewidth}{lX}
\hline
Field  & Description \\
\hline
\var{highlight_colour} &  
	Colour for the ''highlight'' part of a border border. \\
\var{shadow_colour}    &  
	Colour for the ''highlight'' part of a border border. \\
\var{foreground_colour} &  
	Colour for the normal drawing operations, e.g. text. \\
\var{background_colour} &  
	Window background colour (unless transparency is enabled) and
	background colour boxes. \\
\var{padding_colour} &  
	Colour for the ''padding'' part of a border border. Set to 
	\var{background_colour} if unset. \\
\end{tabularx}


\subsubsection{Borders and widths}

All other fields below except \var{border_style} are non-negative integers
indicating a number of pixels.

\begin{tabularx}{\linewidth}{lX}
\hline
Field  & Description \\
\hline
\var{border_style} & A string indicating the style of border; one of
		     elevated/inlaid/ridge/groove as seen in the
		     above sketch. \\
\var{highlight_pixels} &  
	Width of the highlight part of the border in pixels. \\
\var{shadow_pixels}    &  
	Width of the shadow part of the border in pixels. \\
\var{padding_pixels} &  
	Width of the padding part of the border in pixels. \\
\var{spacing} &
	Space to be left between all kinds of boxes. \\
\end{tabularx}


\subsubsection{Text}

\begin{tabularx}{\linewidth}{lX}
\hline
Field  & Description \\
\hline
\var{font} & Font to be used in text-drawing operations; standard X font
	     name. \\
\var{text_align} & How text is to be aligned in text boxes/tabs; one of
		   the strings left/right/center. \\
\end{tabularx}


\subsubsection{Miscellaneous}


\begin{tabularx}{\linewidth}{lX}
\hline
Field  & Description \\
\hline
\var{transparent_background} & Should windows' that use this style
	background be transparent? true/false. \\
\var{based_on} & The name of a previously defined style that this
	style should be based on. \\
\end{tabularx}


\subsubsection{Substyles}

As discussed in previous sections, styles may have substyles to e.g.
indicate different states of the object being drawn. The ''de'' engine
limits what can be configured in substyles to the set of colours in the
first table above, but also specifically interprets for the main style
\code{frame-tab} the substyles \code{*-*-tagged} and \code{*-*-*-dragged}
by, respectively, drawing a right angle shape at the top right corner 
of a tab and by shading the tab with a stipple pattern.

Substyles are defined with the function \fnref{de_substyle} within the
table defining the main style. The parameters to this function are 
similar to those of \fnref{de_define_style}.

\begin{verbatim}
de_define_style("some-style", {
   ...
   de_substyle("some-substyle", {
      ...
   }),
   ...
})
\end{verbatim}


\subsection{An example}

The following shortened segment from \file{look-cleanviolet.lua}
should help to clarify the mattesr discussed in the previous
subsection.

\begin{verbatim}
de_define_style("*", {
    -- Gray background
    highlight_colour = "#eeeeee",
    shadow_colour = "#eeeeee",
    background_colour = "#aaaaaa",
    foreground_colour = "#ffffff",
    
    shadow_pixels = 1,
    highlight_pixels = 1,
    padding_pixels = 1,
    spacing = 0,
    border_style = "elevated",
    
    font = "-*-helvetica-medium-r-normal-*-12-*-*-*-*-*-*-*",
    text_align = "center",
})

de_define_style("frame-tab", {
    based_on = "*",
    
    de_substyle("active-selected", {
        -- Violet tab
        highlight_colour = "#aaaacc",
        shadow_colour = "#aaaacc",
        background_colour = "#666699",
        foreground_colour = "#eeeeee",
    }),

    --  More substyles would follow ...
})
\end{verbatim}


Hopefully that's enough to get you started in writing new style
configuration files for Ion. When in doubt, study the existing
style configuration files.

