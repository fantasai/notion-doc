
\begin{function}
    \index{clientwin-do-manage-alt@\code{clientwin_do_manage_alt}}
    \hookname{clientwin_do_manage_alt}
    \hookparams{(WClientWin, table)}
    \begin{funcdesc}
      Called when we want to manage a new client window.
      The table argument contains the following fields:
      
      \begin{tabularx}{\linewidth}{llX}
          \tabhead{Field & Type & Description}
          \var{switchto} & bool & Do we want to switch to the client window. \\
          \var{jumpto} & bool & Do we want to jump to the client window. \\
          \var{userpos} & bool & Geometry set by user. \\
          \var{dockapp} & bool & Client window is a dock-app. \\
          \var{maprq} & bool & Map request (and not initialisation scan). \\
          \var{gravity} & number & Window gravity. \\
          \var{geom} & table & Requested geometry; \var{x}, \var{y}, \var{w}, \var{h}.\\
          \var{tfor} & WClientWin & Transient for window.
      \end{tabularx}

      This hook is not called in protected mode and can be used for
      arbitrary placement policies (deciding in which workspace a new
      \type{WClientWin} should go). In this case, you can call
\begin{verbatim}
reg:attach(cwin)
\end{verbatim}
      where \var{reg} is the region where the window should go, and
      \var{cwin} is the first argument of the function added to the
      hook.
    \end{funcdesc}
\end{function}


\begin{function}
    \index{clientwin-mapped-hook@\code{clientwin_mapped_hook}}
    \hookname{clientwin_mapped_hook}
    \hookparams{WClientWin}
    \begin{funcdesc}
      Called when we have started to manage a client window.
    \end{funcdesc}
\end{function}


\begin{function}
    \index{clientwin-property-change-hook@\code{clientwin_property_change_hook}}
    \hookname{clientwin_property_change_hook}
    \hookparams{(WClientWin, integer)}
    \begin{funcdesc}
      Called when the property identified by the parameter atom id
      (integer) has changed on a client window.
    \end{funcdesc}
\end{function}


\begin{function}
    \index{clientwin-unmapped-hook@\code{clientwin_unmapped_hook}}
    \hookname{clientwin_unmapped_hook}
    \hookparams{number}
    \begin{funcdesc}
      Called when we no longer manage a client window. The parameter
      is the X ID of the window; see \fnref{WClientWin.xid}.
    \end{funcdesc}
\end{function}


\begin{function}
    \index{frame-managed-changed-hook@\code{frame_managed_changed_hook}}
    \hookname{frame_managed_changed_hook}
    \hookparams{table}
    \begin{funcdesc}
      Called when there are changes in the objects managed by a frame
      or their order. The table parameter has the following fields:

      \begin{tabularx}{\linewidth}{llX}
          \tabhead{Field & Type & Description}
          \var{reg} & WFrame & The frame in question \\
          \var{mode} & string & \var{"switchonly"}, \var{"reorder"},
                                \var{"add"} or \var{"remove"} \\
          \var{sw} & bool & Switch occurred \\
          \var{sub} & WRegion & The managed region (primarily) affected \\
      \end{tabularx}
    \end{funcdesc}
\end{function}


\begin{function}
    \index{ioncore-sigchld-hook@\code{ioncore_sigchld_hook}}
    \hookname{ioncore_sigchld_hook}
    \hookparams{integer}
    \begin{funcdesc}
      Called when a child process has exited. The parameter
      is the PID of the process.
    \end{funcdesc}
\end{function}

\begin{function}
    \index{ioncore-deinit-hook@\code{ioncore_deinit_hook}}
    \hookname{ioncore_deinit_hook}
    \hookparams{()}
    \begin{funcdesc}
      Called when Ion is deinitialising and about to quit.
    \end{funcdesc}
\end{function}

%ioncore_handle_event_alt -- not available to lua side

\begin{function}
    \index{ioncore-post-layout-setup-hook@\code{ioncore_post_layout_setup_hook}}
    \hookname{ioncore_post_layout_setup_hook}
    \hookparams{()}
    \begin{funcdesc}
      Called when Ion has done all initialisation and is almost ready to
      enter the main-loop, except no windows are yet being managed.
    \end{funcdesc}
\end{function}


\begin{function}
    \index{ioncore-snapshot-hook@\code{ioncore_snapshot_hook}}
    \hookname{ioncore_snapshot_hook}
    \hookparams{()}
    \begin{funcdesc}
      Called to signal scripts and modules to save their state (if any).
    \end{funcdesc}
\end{function}


\begin{function}
    \index{tiling-placement-alt@\code{tiling_placement_alt}}
    \hookname{tiling_placement_alt}
    \hookparams{table}
    \begin{funcdesc}
      Called when a client window is about to be managed by a \type{WTiling}
      to allow for alternative placement policies. The table has the
      following fields:
      \begin{tabularx}{\linewidth}{llX}
          \tabhead{Field & Type & Description}
          \var{tiling} & \type{WTiling} & The tiling \\
          \var{reg} & \type{WRegion} & The region (always a WClientWin at 
              the moment) to be placed \\
          \var{mp} & \type{table} & This table contains the same fields as
            the parameter of \fnref{clientwin_do_manage_alt} \\
          \var{res_frame} & \type{WFrame} & A successful handler should 
            return the target frame here. \\
      \end{tabularx}
      This hook is just for placing within a given workspace after the
      workspace has been decided by the default workspace selection
      policy. It is called in protected mode. For arbitrary placement
      policies, \fnref{clientwin_do_manage_alt} should be used; it
      isn't called in protected mode,
    \end{funcdesc}
\end{function}


\begin{function}
    \index{region-do-warp-alt@\code{region_do_warp_alt}}
    \hookname{region_do_warp_alt}
    \hookparams{WRegion}
    \begin{funcdesc}
      This alt-hook exist to allow for alternative pointer warping
      implementations.
    \end{funcdesc}
\end{function}


\begin{function}
    \index{screen-managed-changed-hook@\code{screen_managed_changed_hook}}
    \hookname{screen_managed_changed_hook}
    \hookparams{table}
    \begin{funcdesc}
      Called when there are changes in the objects managed by a screen
      or their order. The table parameter is similar to that of
      \fnref{frame_managed_changed_hook}.
    \end{funcdesc}
\end{function}


\begin{function}
    \index{region-notify-hook@\code{region_notify_hook}}
    \hookname{region_notify_hook}
    \hookparams{(WRegion, string)}
    \begin{funcdesc}
      Signalled when something (minor) has changed in relation to 
      the first parameter region. The string argument gives the
      change:
      
      \begin{tabularx}{\linewidth}{lX}
          \tabhead{String & Description}
          \code{deinit} & The region is about to be deinitialised. \\
          \code{activated} & The region has received focus. \\
          \code{inactivated} & The region has lost focus. \\
          \code{activity} & There's been activity in the region itself. \\
          \code{sub_activity} & There's been activity in some sub-region. \\
          \code{name} & The name of the region has changed. \\
          \code{unset_manager} & The region no longer has a manager. \\
          \code{set_manager} & The region now has a manager. \\
          \code{tag} & Tagging state has changed. \\
          \code{pseudoactivated} & The region has become pseudo-active
                      (see below). \\
          \code{pseudoinactivated} & The region is no longer pseudo-active. \\
      \end{tabularx}

      A region is pseudo-active, when a) it is itself not active (does
      not not have the focus, and may not even have a window that could
      have it), but b) some region managed by it is active.
    \end{funcdesc}
\end{function}


