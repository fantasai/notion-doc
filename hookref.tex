
\begin{function}
    \index{clientwin-do-manage-alt@\code{clientwin_do_manage_alt}}
    \hookname{clientwin_do_manage_alt}
    \hookparams{(WClientWin, table)}
    \begin{funcdesc}
      Called when we want to manage a new client window.
      The table argument contains the following fields:
      
      \begin{tabularx}{\linewidth}{llX}
          \tabhead{Field & Type & Description}
          \var{switchto} & bool & Do we want to switch to the client window. \\
          \var{jumpto} & bool & Do we want to jump to the client window. \\
          \var{userpos} & bool & Geometry set by user. \\
          \var{dockapp} & bool & Client window is a dockapp. \\
          \var{maprq} & bool & Map request (and not initialisation scan). \\
          \var{gravity} & number & Window gravity. \\
          \var{geom} & table & Requested geometry; \var{x}, \var{y}, \var{w}, \var{h}.\\
          \var{tfor} & WClientWin & Transient for window.
      \end{tabularx}

      This hook is not called in protected mode and can be used for
      arbitrary placement policies (deciding in which workspace a new
      \type{WClientWin} should go). In this case, you can call
\begin{verbatim}
reg:attach(cwin)
\end{verbatim}
      where \var{reg} is the region where the window should go, and
      \var{cwin} is the first argument of the function added to the
      hook.
    \end{funcdesc}
\end{function}


\begin{function}
    \index{clientwin-mapped-hook@\code{clientwin_mapped_hook}}
    \hookname{clientwin_mapped_hook}
    \hookparams{WClientWin}
    \begin{funcdesc}
      Called when we have started to manage a client window.
    \end{funcdesc}
\end{function}


\begin{function}
    \index{clientwin-property-change-hook@\code{clientwin_property_change_hook}}
    \hookname{clientwin_property_change_hook}
    \hookparams{(WClientWin, integer)}
    \begin{funcdesc}
      Called when the property identified by the parameter atom id
      (integer) has changed on a client window.
    \end{funcdesc}
\end{function}


\begin{function}
    \index{clientwin-unmapped-hook@\code{clientwin_unmapped_hook}}
    \hookname{clientwin_unmapped_hook}
    \hookparams{number}
    \begin{funcdesc}
      Called when we no longer manage a client window. The parameter
      is the X ID of the window; see \fnref{WClientWin.xid}.
    \end{funcdesc}
\end{function}


\begin{function}
    \index{frame-managed-changed-hook@\code{frame_managed_changed_hook}}
    \hookname{frame_managed_changed_hook}
    \hookparams{table}
    \begin{funcdesc}
      Called when there are changes in the objects managed by a frame
      or their order. The table parameter has the following fields:

      \begin{tabularx}{\linewidth}{llX}
          \tabhead{Field & Type & Description}
          \var{reg} & WFrame & The frame in question \\
          \var{mode} & string & \var{"switchonly"}, \var{"reorder"},
                                \var{"add"} or \var{"remove"} \\
          \var{sw} & bool & Switch occured \\
          \var{sub} & WRegion & The managed region (primarily) affected \\
      \end{tabularx}
    \end{funcdesc}
\end{function}


\begin{function}
    \index{ioncore-sigchld-hook@\code{ioncore_sigchld_hook}}
    \hookname{ioncore_sigchld_hook}
    \hookparams{integer}
    \begin{funcdesc}
      Called when a child process has exited. The parameter
      is the PID of the process.
    \end{funcdesc}
\end{function}

\begin{function}
    \index{ioncore-deinit-hook@\code{ioncore_deinit_hook}}
    \hookname{ioncore_deinit_hook}
    \hookparams{()}
    \begin{funcdesc}
      Called when Ion is deinitialising and about to quit.
    \end{funcdesc}
\end{function}

%ioncore_handle_event_alt -- not available to lua side

\begin{function}
    \index{ioncore-post-layout-setup-hook@\code{ioncore_post_layout_setup_hook}}
    \hookname{ioncore_post_layout_setup_hook}
    \hookparams{()}
    \begin{funcdesc}
      Called when Ion has done all initialisation and is almost ready to
      enter the mainloop, except no windows are yet being managed.
    \end{funcdesc}
\end{function}


\begin{function}
    \index{ioncore-snapshot-hook@\code{ioncore_snapshot_hook}}
    \hookname{ioncore_snapshot_hook}
    \hookparams{()}
    \begin{funcdesc}
      Called to signal scripts and modules to save their state (if any).
    \end{funcdesc}
\end{function}


\begin{function}
    \index{tiling-placement-alt@\code{tiling_placement_alt}}
    \hookname{tiling_placement_alt}
    \hookparams{table}
    \begin{funcdesc}
      Called when a client window is about to be managed by a \type{WTiling}
      to allow for alternative placement policies. The table has the
      following fields:
      \begin{tabularx}{\linewidth}{llX}
          \tabhead{Field & Type & Description}
          \var{tiling} & \type{WTiling} & The tiling \\
          \var{reg} & \type{WRegion} & The region (always a WClientWin at 
              the moment) to be placed \\
          \var{mp} & \type{table} & This table contains the same fields as
            the parameter of \fnref{clientwin_do_manage_alt} \\
          \var{res_frame} & \type{WFrame} & A succesfull handler should 
            return the target frame here. \\
      \end{tabularx}
      This hook is just for placing within a given workspace after the
      workspace has been decided by the default workspace selection
      policy. It is called in protected mode. For arbitrary placement
      policies, \fnref{clientwin_do_manage_alt} should be used; it
      isn't called in protected mode,
    \end{funcdesc}
\end{function}


%\begin{function}
%    \index{panews-init-layout-alt@\code{panews_init_layout_alt}}
%    \hookname{panews_init_layout_alt}
%    \hookparams{table}
%    \begin{funcdesc}
%      Called to initialise panews layout. The table parameter has
%      initially a single field \var{ws} pointing to the workspace,
%      but the succesfull handler should set the field \var{layout}
%      to a proper layout (as those saved in the layout savefiles).
%    \end{funcdesc}
%\end{function}


\begin{function}
    \index{panews-make-placement-alt@\code{panews_make_placement_alt}}
    \hookname{panews_make_placement_alt}
    \hookparams{table}
    \begin{funcdesc}
      Called to make a placement on panews. The parameter table has
      the following fields:
      
      \begin{tabularx}{\linewidth}{llX}
          \tabhead{Field & Type & Description}
          \var{ws} & \type{WPaneWS} & The workspace \\
          \var{frame} & \type{WFrame} & A frame initially allocated for the
              region to be placed \\
          \var{reg} & \type{WRegion} & The region to be placed \\
          \var{specifier} & \type{WRegion} & For drag\&drop on handling empty areas\\
      \end{tabularx}
      
      The handler should set some of these fields on success:

      \begin{tabularx}{\linewidth}{llX}
          \tabhead{Field & Type & Description}
          \var{res_node} & \type{WSplit} & Target split \\
          \var{res_config} & \type{WFrame} &  New configuration for it, unless
              \type{WSplitRegion} \\
          \var{res_w} & integer & New width for target split (optional) \\
          \var{res_h} & integer & New height for target split (optional) \\
      \end{tabularx}
    \end{funcdesc}
\end{function}


\begin{function}
    \index{region-activated-hook@\code{region_activated_hook}}
    \hookname{region_activated_hook}
    \hookparams{WRegion}
    \begin{funcdesc}
      Signalled when a region or one of its children has received the focus.
    \end{funcdesc}
\end{function}


\begin{function}
    \index{region-activity-hook@\code{region_activity_hook}}
    \hookname{region_activity_hook}
    \hookparams{WRegion}
    \begin{funcdesc}
      This hook is triggered when the activity flag of the parameter 
      region has been changed.
    \end{funcdesc}
\end{function}


\begin{function}
    \index{region-do-warp-alt@\code{region_do_warp_alt}}
    \hookname{region_do_warp_alt}
    \hookparams{WRegion}
    \begin{funcdesc}
      This alt-hook exist to allow for alternative pointer warping
      implementations.
    \end{funcdesc}
\end{function}


\begin{function}
    \index{region-inactivated-hook@\code{region_inactivated_hook}}
    \hookname{region_inactivated_hook}
    \hookparams{WRegion}
    \begin{funcdesc}
      Signalled when the focus has moved from the parameter region or
      one of its children to a non-child region of the parameter region.
    \end{funcdesc}
\end{function}


\begin{function}
    \index{screen-managed-changed-hook@\code{screen_managed_changed_hook}}
    \hookname{screen_managed_changed_hook}
    \hookparams{table}
    \begin{funcdesc}
      Called when there are changes in the objects managed by a screen
      or their order. The table parameter is similar to that of
      \fnref{frame_managed_changed_hook}.
    \end{funcdesc}
\end{function}
