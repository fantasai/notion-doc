\section{The statusbar}
\label{sec:statusbar}

The \file{mod\_statusbar} module provides a statusbar that adapts to 
layouts of tilings, using only the minimal space needed. Ion only 
supports one adaptive ``status display'' object per screen, so this
statusbar is mutually exclusive with the embedded mode of \file{mod\_dock} 
docks. 

The statusbar is configured in \file{cfg\_statusbar.lua}. Typically,
the configuration consists of two steps: creating a statusbar with
\fnref{mod\_statusbar.create}, and then launching the separate
\file{ion-statusd} status daemon process with 
\fnref{mod_statusbar.launch_statusd}. This latter phase is done
automatically, if it was not done by the configuration file, but
the configuration file may pass extra parameters to \file{ion-statusd}
monitors. (See Section \ref{sec:statusd} for more information on
writing \file{ion-statusd} monitors.)

A typical \file{cfg\_statusbar.lua} configuration might look as follows:


\begin{verbatim}
-- Create a statusbar
mod_statusbar.create{
    screen = 0,     -- First screen, 
    pos = 'bl',     -- bottom left corner
    systray = true, -- Swallow systray windows

    -- The template
    template = "[ %date || load:% %>load || mail:% %>mail_new/%>mail_total ]"
               .. " %filler%systray",
}

-- Launch ion-statusd. 
mod_statusbar.launch_statusd{
    -- Date meter
    date={
        -- ISO-8601 date format with additional abbreviated day name
        date_format='%a %Y-%m-%d %H:%M',
    },      
}
\end{verbatim}


\subsection{The template}

The template specifies what is shown on the statusbar; for information
on the other options to \fnref{mod_statusbar.create}, see the reference. 
Strings of the form \verb!%spec! tokens specially interpreter by
the statusbar; the rest appears verbatim. The \code{spec} typically
consists of the name of the value/meter to display (beginning with a latin
alphabet), but may be preceded by an alignment specifier and a number
specifying the minimum width. The alignmenter specifiers are: \code{>}
for right, \code{<} for left,  and \code{|} for centering. Additionally,
space following \verb!%! (that is, the string \verb!"% "!), adds
``stretchable space'' at that point. The special string \verb!%filler!
may be used to flush the rest of the template to the right end of 
the statusbar. 

The stretchable space works as follows: \file{mod\_statusbar} remembers
the widest string (in terms of graphical presentation) that it has
seen for each meter, unless the width has been otherwise constrained.
If there is stretchable space in the template, it tries to make the
meter always take this much space, by stretching any space found in
the direction indicated by the alignment specifier: the opposite
direction for left or right alignment, and both for centering.

\subsection{The systray}

The special \verb!%systray! monitor indicates where to place system
tray windows. There may be multiple of these. KDE-protocol system tray
icons are placed there automatically unless disabled with the
\var{systray} option. Otherwise the \var{statusbar} winprop may
be used.

\subsection{Monitors}

The part before the first
underscore of each monitor name, descripts the scripts/plugin/module
that provides the meter, and any configuration should be passed
in the a corresponding sub-table \fnref{mod_statusbar.launch_statusd}.
Ion comes with date, load and mail (for plain old mbox) 
\file{ion-statusd} monitor scripts. More may be obtained from 
the scripts repository \cite{scripts}. These included scripts 
provide the following monitors and their options

\subsubsection{Date}

Options: \var{date_format}: The date format in as seen above, 
in the usual \code{strftime} format. \code{formats}: table of
formats for additional date monitors, the key being the name
of the monitor (without the \code{"date_"} prefix).

Monitors: \code{date} and other user-specified ones with the
\code{date_} prefix.


\subsubsection{Load}

Options: \var{update_interval}: Update interval in milliseconds
(default 10s). \var{important_threshold}: Threshold above which 
the load is marked as important (default 1.5), so that the 
drawing engine may be suitably hinted. \var{critical_threshold}: 
Threshold above which  the load is marked as critical (default 4.0).


Monitors: \code{load} (for all three values), 
\code{load_1min}, \code{load_5min} and \code{load_15min}.


\subsubsection{Mail}

Options: \var{update_interval}: Update interval in milliseconds
(default 1min). \var{mbox}: mbox-format mailbox location
(default \verb!$MAIL!). 
\var{files}: list of additional mailboxes, the key giving the 
name of the monitor.

Monitors: \code{mail_new}, \code{mail_unread},
\code{mail_total}, and corresponding
\code{mail_*_new}, \code{mail_*_unread}, and \code{mail_*_total}
for the additional mailboxes (\code{*} varying).
