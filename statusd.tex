
\section{Writing \command{ion-statusd} monitors}
\label{sec:statusd}

All statusbar meters that do not monitor the internal state of Notion should
go in the separate \command{ion-statusd} program. 

Whenever the user requests a meter \codestr{\%foo} or \codestr{\%foo\_bar} to 
be  inserted in a statusbar, \file{mod\_statusbar} asks \command{ion-statusd} 
to load \fnref{statusd_foo.lua} on its search path (same as that for Notion-side 
scripts). This script should then supply all meters with the initial part
\codestr{foo}.

To provide this value, the script should simply call \code{statusd.inform}
with the name of the meter and the value as a string.
Additionally the script should provide a 'template' for the meter to
facilitate expected width calculation by \file{mod\_statusbar}, and
may provide a 'hint' for colour-coding the value. The interpretation
of hints depends on the graphical style in use, and currently the
stock styles support the \codestr{normal}, \codestr{important} and 
\codestr{critical} hints.


In our example of the 'foo monitor', at script initialisation we might broadcast
the template as follows:

\begin{verbatim}
statusd.inform("foo_template", "000")
\end{verbatim}

To inform \file{mod\_statusbar} of the actual value of the meter and
indicate that the value is critical if above 100, we might write the
following function:

\begin{verbatim}
local function inform_foo(foo)
    statusd.inform("foo", tostring(foo))
    if foo>100 then
        statusd.inform("foo_hint", "critical")
    else
        statusd.inform("foo_hint", "normal")
    end
end    
\end{verbatim}
    
To periodically update the value of the meter, we must use timers.
First we must create one:

\begin{verbatim}
local foo_timer=statusd.create_timer()
\end{verbatim}

Then we write a function to be called whenever the timer expires.
This function must also restart the timer.

\begin{verbatim}
local function update_foo()
    local foo= ... measure foo somehow ...
    inform_foo(foo)
    foo_timer:set(settings.update_interval, update_foo)
end
\end{verbatim}

Finally, at the end of our script we want to do the initial
measurement, and set up timer for further measurements:

\begin{verbatim}
update_foo()
\end{verbatim}


If our scripts supports configurable parameters, the following code
(at the beginning of the script) will allow them to be configured in
\file{cfg\_statusbar.lua} and passed to the status daemon and our script:

\begin{verbatim}
local defaults={
    update_interval=10*1000, -- 10 seconds
}
                
local settings=table.join(statusd.get_config("foo"), defaults)
\end{verbatim}
