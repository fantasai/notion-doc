
\chapter{Introduction}

This document is a sort of ''advanced user'' manual for Ion, the X11
window manager. It is an attempt attempt at documenting what is in Ion's
configuration files, how to configure Ion by simple modifications to these
files and how to write more complex extensions in Lua, the lightweight
configuration and scripting language used by Ion. 

If the reader is new to Lua, he is advised to first glance at the Lua
manual at \url{http://www.lua.org/docs.html} and/or perhaps some helpfull
lua-users wiki pages at \url{http://lua-users.org/wiki/}, in particular 
the tutorial pages there (\url{http://lua-users.org/wiki/LuaTutorial}).

Back in this document, first, in chapter \ref{chap:prelim} Ion's object 
and class hierarchies are explained. While it might not at first occur 
that knowing such things would be necessary to \emph{configure} a program,
this material is essential because of the object-oriented nature of most 
of Ion's scripting interface.

The new user, fed up with the default key bindings and eager to just
quickly configure Ion to his liking may question the reasons for 
exposing the ''heavy'' internal OO structure in the scripting and
configuration interface and I'm no the one to  blame him for that. 
Sure it would be easier for new users to  configure Ion to the liking
if a consistently simpler configuration interface was provided. Such an
interface would, however, also be far more limited or at least make 
writing extensions more complicated and the advantages from using a 
real scripting language would be partly lost. One more advantage from
a rich scripting and configuration interface is that it allows 
implementing scripts to read alternate configuration file formats,
one that could be e.g. modified by external configuration tools. 

In chapter \ref{chap:config} the very basic Ion configuration know-how
is provided. The different configuration files and their locations
are explained and instructions are given to allow the reader to
configure bindings and so-called ''winprops''.  Chapter \ref{chap:gr}
explains the notion of drawing engines and graphical styles and how to
write new looks for Ion and more advanced aspects of Ion's scripting 
interface are documented in chapter \ref{chap:tricks} (a work in 
progress).

Finally, most of the functions provided by Ion's scripting interface
are listed and documented in the Function reference in 
chapter \ref{sec:exports}.


