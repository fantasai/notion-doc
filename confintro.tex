
\chapter{Introduction}

This document is an ''advanced user'' manual for the X11 window manager
Ion, version 3. It is an attempt at documenting things that go into Ion's
configuration files, how to configure Ion by simple modifications 
to these files and how to write more complex extensions in Lua, the 
lightweight configuration and scripting language used by Ion. 

Readers unfamiliar with Lua might first want to first glance at some 
Lua documentation at

\begin{center}
    \url{http://www.lua.org/docs.html}, or \\
    \url{http://lua-users.org/wiki/LuaTutorial},
\end{center}

although this should not be strictly necessary for basic modifications
of configuration files for anyone with at least some familiarity with
programming languages.

Back in this document, first in chapter \ref{chap:prelim} some key
concepts and relations are explained. These include the module system,
and Ion's object (or ``region'') and class hierarchies. While it may
not be necessary to study the latter for basic copy-paste modifications
of configuration files -- for that you should not really need this
manual either -- it is, however, essential to for more extensive
customisation, due to the semi-object-oriented nature of most of
Ion's scripting interface. Knowing the different object types also
helps dealing with the different binding ``contexts'' (see
Section \ref{sec:bindings}) that to some extent mirror these classes.

The new user, fed up with the default key bindings and eager to 
just quickly configure Ion to his liking, may therefore just want
to skip to Chapter \ref{chap:config}, and attempt to work from therefore.
That chapter provides the very basic Ion configuration know-how
is provided: all the different configuration files and their locations
are explained, instructions are given to allow the reader to
configure bindings and so-called ''winprops'', and the statusbar
templates are also explained. 

Next, Chapter \ref{chap:gr} explains the notion of drawing engines
and graphical styles and how to write new looks for Ion. More advanced
aspects of Ion's scripting interface are documented in Chapter 
\ref{chap:tricks}. 
Finally, most of the functions provided by Ion's scripting interface
are listed and documented in the Function reference in Chapter
\ref{sec:exports}. At the end of the document an alphabetical
listing of all these functions may be found.

