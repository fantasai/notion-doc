\section{Object system implementation}

First, to get things clear, what are considered objects here are C
structures containing a properly initialised \type\indextype{WObj}
structure defined in \file{ioncore/obj.h} as the first element (or the
first element of the structure which is the first element and so on which
gives rise to inheritance). The \type{WObj} structure contains a pointer
to a \type{WObjDescr}\indextype{WObjDescr} class type info structure and
a list of so called ``watches''. The \type{WObjDescr} structure simply
lists the class name, a table of dynamic functions and a pointer to
deinitialisation function (or ``destructor'').

\subsection{Safe references}

NOTE: This subsection appears out-of-date, it seems the WWatch infrastructure
was removed from Ion3.

Notion does not do any reference counting, garbage collecting or other
fancy things related to automatic safe freeing of objects with its
simplistic object system. Instead special watches (the \type{WWatch}
\indextype{WWatch} structure) may be used to create safe references to
objects that might be destroyed during the time the specific pointer is
needed. When an object is destroyed, its list of watches is processed,
setting the pointers in the watches to NULL and the watch handlers for
each watch are called. 

\subsection{Dynamic dispatch}

To override a method means to reimplement it specifically for a given subtype.
This is used in Notion to allow subclasses to override of their superclasses' 
methods.

Overridable methods are specified as \code{DYNFUN}, for example 
\fnref{region_fitrep} in \file{ioncore/region.h}:

\begin{verbatim}
DYNFUN bool region_fitrep(WRegion *reg, WWindow *par, const WFitParams *fp);
\end{verbatim}

Dynamic functions can be called with \code{CALL_DYN} or \code{CALL_DYN_RET}.
When a subclass desires to override a dynamic function, it does so by 
specifying a \type{DynFunTab}\indextype{DynFunTab}. For example, the function
table for \type{WGroup} overrides \fnref{region_fitrep} with 
\fnref{group_fitrep}:

\begin{verbatim}
static DynFunTab group_dynfuntab[]={
    {(DynFun*)region_fitrep,
     (DynFun*)group_fitrep},
    ...
    END_DYNFUNTAB
};
EXTL_EXPORT
IMPLCLASS(WGroup, WRegion, group_deinit, group_dynfuntab);
\end{verbatim}

Whenever \fnref{region_fitrep} is called on a region that is actually a 
\type{WGroup}, \fnref{group_fitrep} is called instead.
