\section{C coding style}

If you want to submit patches to Ion, you MUST follow my coding 
style, even if you think it is the root of all evil. We don't want
the code to be an incomprehensible mess of styles and I have better
things to do than fix other people's style to match mine. The style
should be obvious by studying the source, but here's a list of some
things to take note of.

\subsection{Whitespace}

\begin{itemize}
  \item Indentations of 4 with \emph{tab size=4}. 
    
  \item No extra spaces between operators, delimiters etc. except
    \begin{itemize}
      \item around logical and, or (\code{&&}, \code{||})
      \item around the conditional \code{a ? b : c}
      \item after commas and semicolons
    \end{itemize}  
    In my opinion this helps pointing out arithmetic or other
    expressions within logical expressions or parameter lists.
    
  \item All kinds of labels are out-tended to the level of the higher
    level block. For example:

\begin{verbatim}    
void foo()
{
again:
    switch(asdf){
    case 1:
        ...
        break;
    default:
        ...
        break;
    }
}
\end{verbatim}
\end{itemize}

\subsection{Braces}

\begin{itemize}
\item Opening brace is at the end of the line, except in function
    bodies, where it is at the beginning of the line following
    the definition.

\item Never put the body of a control statement on the same line
    with the statement (e.g. \verb!if(foo){ bar() }!).

For example, the block
\begin{verbatim}    
void foo(int a, int b)
{
    if(a==b && c+d==e){
        ...
    }
}
\end{verbatim}

has correct style while the block

\begin{verbatim}   
void foo(int a,int b) {
    if (a == b && c + d == e) {
        ...
    }
}
\end{verbatim}

does not.

  \item The \code{else} keyword follows immediately after the closing brace of
    previous \code{if}, if any. (This might change so I don't care if you put
    it on the next line.)
    
  \item I have used the convention that control statement bodies containing
    a single statement do not need braces around the block if, in case of
    the \code{if} all the blocks in  \code{if ...  else if ... else}
    contain just one statement. If you want to, just use braces in every 
    case.
\end{itemize}

\subsection{Names}

\begin{itemize}
  \item Function and variable names only have lower case letters. Type
    names are in mixed case while constants and macros (\code{#define}s)
    are in upper case letters.
\end{itemize}

\subsection{Miscellaneous}

\begin{itemize}
  \item In the definition of a pointer variable, the asterisk is attached
    to the variable name: \code{char *s;}. (One could claim this an 
    exception to the second rule.)
    
  \item You might optionally want to use Jed's foldings to group blocks
    of related code in a file to keep it organized:

\begin{verbatim}    
/*{{{ Many related functions */
	
void code()
{
    ...	
}

...

/*}}}*/
\end{verbatim}
\end{itemize}

I think that's mostly it. Study the source when in doubt.
