
\xchapter{Preliminaries: The Ion class and object hierarchies}
\label{chap:prelim}

Although Ion does not have a truely object-oriented design -- and I,
the author of this document and Ion wouldn't even like such a design -- 
it does have a primitive object oriented design for ''screen objects''.
It is essential for the module writer to learn this object system,
but it should be usefull for the person who just wants to configure
key bindings to learn at least something about Ion's object hierarchy.
When you know the class and object hierarchies, you know how and where
the available functions can be used and how to construct more complex
scripts. Therefore I suggest that you read this short chapter or at
least the summary at the end of it before skipping on to the
more interesting things and even trying to write custom binding
configuration files.


\xsection{Class hierarchy}

One of the most important principles of object-oriented design methodology
is inheritance -- roughly how classes (objects are instances of classes)
extend on others' features. Inheritance gives rise to class hierarchy.
In the case of single-inheritance this hierarchy can be expressed as a
tree where the object at the root is inherited by all others below it
and so on.

Figure \ref{fig:classhierarchy} lists out the Ion class hierarchy and below
we explain what the classes implement. If you are wondering what the 
''(xyz module)'' strings in the figure are, they're related to the 
division of Ion into a core binary and multiple modules that implement
optional features. Module support also allows implementing  C-side
extensions to Ion without changes to Ion's code or the need of 
recompiling if the operating system has decent support for dynamic 
linking.

\begin{figure}
\begin{htmlonly}
\docode % latex2html kludge
\end{htmlonly}
\begin{verbatim}
    WObj
     |
     |-->WRegion
     |    |
     |    |-->WClientWin
     |    |
     |    |-->WWindow
     |    |    |
     |    |    |-->WRootWin
     |    |    |
     |    |    |-->WMPlex
     |    |    |    |
     |    |    |    |-->WScreen
     |    |    |    |
     |    |    |    |-->WGenFrame
     |    |    |         |
     |    |    |         |-->WIonFrame (ionws module)
     |    |    |         |
     |    |    |         |-->WFloatFrame (floatws module)
     |    |    |
     |    |    |-->WInput (query module)
     |    |         |
     |    |         |-->WEdln (query module)
     |    |         |
     |    |         |-->WMessage (query module)
     |    |
     |    |-->WGenWS
     |         |
     |         |-->WIonWS (ionws module)
     |         |
     |         |-->WFloatWS (floatws module)
     |
     |-->WWsSplit (ionws)
\end{verbatim}
\caption{Ion class hierarchy. The string in parenthesis indicates
  the module in which this class is implemented if not in Ioncore.}
\label{fig:classhierarchy}
\end{figure}

The core classes:

\begin{description}
  \item[\type{WObj}]\indextype{WObj}
    Is the base of Ion's object system.
    
  \item[\type{WRegion}]\indextype{WRegion}
    is the base class for everything corresponding to something on the
    screen. Each object of type \type{WRegion} has a size and  position
    relative to the parent \type{WRegion}. While a great part of Ion 
    operates on these instead of more specialised classes, \type{WRegion}
    is a ''virtual''  base class in that there are no objects of ''pure''
    type \type{WRegion}; all concrete regions are objects of some class 
    that inherits \type{WRegion}.

  \item[\type{WClientWin}\indextype{WClientWin}] is a class for
    client window objects -- the objects that window managers are
    supposed to manage.

  \item[\type{WWindow}]\indextype{WWindow} is the base class for all
    internal objects having an X window associated to them
    (\type{WClientWins} also have X windows associated to them).
    
  \item[\type{WRootWin}]\indextype{WRootWin} is the class for
    root windows\index{root window} of X screens\index{screen!X}.
    Note that an ''X screen'' or root window is not necessarily a
    single  physical screen\index{screen!physical} as a root window
    may be split over multiple screens when multi-head extensions 
    such as Xinerama\index{Xinerama} are used. (Actually there
    can be only one \type{WRootWin} when Xinerama is used.)
	
  \item[\type{WMPlex}] is again a virtual base class for all regions that
    ''multiplex'' other regions. This means that of the regions managed by
    the multiplexer, only one can be displayed at a time. \type{WMPlex}es 
    include screens and the different types of frames.
    
  \item[\type{WScreen}]\indextype{WScreen} is the class for objects
    corresponding to physical screens. Screens may share a root
    window when Xinerama multihead extensions are used as explained
    above.

  \item[\type{WGenFrame}]\indextype{WGenFrame} is another virtual class
    implementing what is common to the concrete frame classes but not
    all \type{WMPlex}es. While most Ion's objects have no graphical 
    presentation, frames are basically the decorations around client
    windows (borders, tabs).

  \item[\type{WGenWS}]\indextype{WGenWS} is a light virtual base class
    for different types of workspaces.
\end{description}


Classes implemented by the ionws module:

\begin{description}
  \item[\type{WIonWS}]\indextype{WIonWS} is the class for the
    usual tiled workspaces.
  \item[\type{WIonFrame}]\indextype{WIonFrame} implements the
    style of frames seen on \type{WIonWS}s.
  \item[\type{WWsSplit}]\indextype{WWsSplit} is an object internal to
    \type{WIonWS} implementation and stores the tree structure of the
    workspaces.
\end{description}


Classes implemented by the floatws module:

\begin{description}
  \item[\type{WFloatWS}]\indextype{WFloatWS} is the class for the
    conventional workspaces where frames and other objects are
    allowed to ''float'' around.
  \item[\type{WFloatFrame}]\indextype{WIonFrame} implements the
    frames seen on \type{WFloatWS}s decorated in the PWM style.
\end{description}


Classes implemented by the query module:

\begin{description}
  \item[\type{WInput}]\indextype{WInput} is a virtual base class for the
    two classes below.
  \item[\type{WEdln}]\indextype{WEdln} is the class for the ''queries'',
    the text inputs that usually appear at bottoms of frames and sometimes
    screens. Quiries are the functional equivalent of ''mini buffers'' in
    many text editors.
  \item[\type{WMessage}]\indextype{WMessage} implements the boxes for 
    warning and other messages that Ion may wish to display to the user. 
    These also usually appear at bottoms of frames.
\end{description}


\xsection{Object hierarchies: \type{WRegion} parents and managers}

Each object of type \type{WRegion} has a parent and possibly a manager
associated to it. The parent\index{parent} for an object is always a 
\type{WWindow} and for \type{WRegion} with an X window (\type{WClientWin},
\type{WWindow}) the parent \type{WWindow} is given by the same relation of
the X windows. For other \type{WRegion}s the relation is not as clear.
There is generally very few restrictions other than the above on the
parent---child relation but the most common is as described in
Figure \ref{fig:parentship}.

\begin{figure}
\begin{htmlonly}
\docode % latex2html kludge
\end{htmlonly}
\begin{verbatim}
    WRootWins
     |
     |-->WScreens
          |
          |-->WIonWS:s and WFloatWS:s
          |
          |-->WClientWins in full screen mode
          |
          |-->WIonFrames and WFloatFrames
               |
               |-->WClientWins, including transients
               |
               |-->a possible WEdln or WMessage
\end{verbatim}
\caption{Most common parent--child relations}
\label{fig:parentship}
\end{figure}

\type{WRegion}s have very little control over their children as a parent.
The manager\index{manager} \type{WRegion} has much more control over its
managed \type{WRegion}s. Managers, for example, handle resize requests,
focusing and displaying of the managed regions. Indeed the manager---managed
relationship gives a better picture of the logical ordering of objects on
the screen. Again, there are generally few limits, but the most common
hierarchy is given in Figure \ref{fig:managership}. Note that sometimes
the parent and manager are the same object and not all objects may have
a manager (e.g. the dock in the dock module at the time of writing this)
but all have a parent--a screen if not anything else.

\begin{figure}
\begin{htmlonly}
\docode % latex2html kludge
\end{htmlonly}
\begin{verbatim}
    WRootWins
     |
     |-->WScreens
          |
          |-->full screen WClientWins
          |    |
          |    |-->transient WClientWins (dialogs)
          |
          |-->WIonWSs
          |    |
          |    |-->WIonFrames
          |         |
          |         |-->WClientWins
          |         |    |
          |         |    |-->transient WClientWins (dialogs)
          |         |
          |         |-->a possible WEdln or WMessage
          |
          |-->WFloatWSs
               |
               |-->WFloatFrames
                    |
                    |-->WClientWins
                    |
                    |-->a possible WEdln or WMessage
\end{verbatim}
\caption{Most common manager--managed relations}
\label{fig:managership}
\end{figure}

Note how the \type{WClientWin}s managed by \type{WFloatFrame}s don't have
transients managed by them. This is because WFloatWSs choose to handle
transients differently (transients are put in separate frames like normal
windows; in the future they should be stacked above the frame containing the
\code{transient_for} window).


\xsection{Summary}

In the standard setup, keeping queries, messages and menus out of
consideration:

\begin{itemize}
  \item The top-level objects that matter are screens and they correspond
    to physical screens. The class for screens is \type{WScreen}.
  \item Screens contain (multiplex) workspaces and full screen client
    windows. Only one of them can be viewed at a time.
    Workspace classes are \type{WIonWS} and \type{WFloatWS} 
    for the two different types of workspaces (and \type{WGenWS}).
  \item Frames are put on workspaces. Frames are the objects with 
    decorations such as tabs and borders.
    Different frame classes are \type{WIonFrame} and \type{WFloatFrame}
    for the two types of frames. The common base class, the features and
    bindings of which both of these inherit is \type{WGenFrame}. 
    
    Workspace of type \type{WIonWS} tile \type{WIonFrame}s while
    workspaces of type \type{WFloatWS} contain \type{WFloatFrames}
    on a more traditional freely-floating fashion.

  \item Frames contain (multiplex) client windows, to each of which
    corresponds a tab in the frame's decoration. Only one client
    window (or other object) can be shown at a time in each frame.
    The class for client windows is \type{WClientWin}.
\end{itemize}

