
\section{Preliminaries: The Ion object system and hierarchy}

\subsection{Implementation basics}

It may be usefull for the scripter too to learn some of Ion's object
system and hierarchy to know how to write proper scripts.

First, to get things clear, what are considered objects here are C
structures containing a properly initialized \type\indextype{WObj}
structure defined in \file{ioncore/obj.c} as the first element (or the
first element of the structure which is the first element and so on which
gives rise to inheritance). The \type{WObj} structure contains a pointer
to a \type{WObjDescr}\indextype{WObjDescr} class type info structure and
a list of so called ''watches''. The \type{WObjDescr} structure simply
lists the class name, a table of dynamic functions and a pointer to
deinitialisation function (or ''destructor'').

Ion does not do any reference counting, garbage collecting or other
fancy things related to automatic safe freeing of objects with its
simplistic object system. Instead special watches (the \type{WWatch}
\indextype{WWatch} structure) may be used to create safe references to
objects that might be destroyed during the time the specific pointer is
needed. When an object is destroyed, its list of watches is processed,
setting the pointers in the watches to NULL and the watch handlers for
each watch are called. 

\subsection{Class hierarchy}

Now that the basics have been covered, Figure \ref{fig:classhierarchy}
lists out the Ion class hierarchy and the classes are explained below:

\begin{figure}
\iftrue
\begin{verbatim}
    WObj
     |
     |-->WRegion
     |    |
     |    |-->WClientWin
     |    |
     |    |-->WWindow
     |    |    |
     |    |    |-->WScreen
     |    |    |
     |    |    |-->WGenFrame
     |    |    |    |
     |    |    |    |-->WIonFrame (ionws.so)
     |    |    |    |
     |    |    |    |-->WFloatFrame (floatws.so)
     |    |    |
     |    |    |-->WInput (query.so)
     |    |         |
     |    |         |-->WEdln (query.so)
     |    |         |
     |    |         |-->WMessage (query.so)
     |    |
     |    |-->WGenWS
     |    |    |
     |    |    |-->WIonWS (ionws.so)
     |    |    |
     |    |    |-->WFloatWS (floatws.so)
     |    |
     |    |-->WViewport
     |
     |-->WWsSplit
\end{verbatim}
\fi
%\includegraphics{objhier.eps}
%\begin{center}
%\input{objhier.latex}
%\end{center}
\caption{Ion class hierarchy}
\label{fig:classhierarchy}
\end{figure}

\begin{description}
  \item[\type{WRegion}\indextype{WRegion}]
    is the base object for everything on the screen
    and most actions done on these. Each WRegion has a size and position
    relative to the parent WRegion. (More on parents and managers
    below).
    \begin{description}
      \item[\type{WClientWin}\indextype{WClientWin}] is an object
	referencing to client windows--what window managers are supposed
	to manage.
	
      \item[\type{WWindow}]\indextype{WWindow} is the base for all internal
	objects having an X window associated to them (\type{WClientWins}
	also have X windows associated to them).
	
      \item[\type{WScreen}]\indextype{WScreen} is the root window of an
	X screen. Note that an X screen is not necessarily a single physical
	screen as it may be split over multiple screens when Xinerama is
	used. The \type{WViewport} objects are supposed to point to the
	areas of the \type{WScreen} that are visible on single physical
	screens (actually there can be only one \type{WScreen} if Xinerama
	is used).
	
      \item[\type{WGenFrame}]\indextype{WGenFrame} is a ''virtual'' (to
	lend the C++ terminology) base class implementing everything in
	common to Ion and PWM (''floating'') style frames and why not others
	if such are ever implemented.
	
      \item[\type{WInput}]\indextype{WInput} is again a virtual base class
	for the different kinds of inputs implemented by the query module
	(the line editor and message display).
	
      \item[\type{WGenWS}]\indextype{WGenWS} like \type{WGenFrame} is a
	virtual base class for the two types of workspaces.
    \end{description}
    
  \item[\type{WWsSplit}]\indextype{WWsSplit} is an object internal to
    \type{WIonWS} implementation.
\end{description}


\subsection{\type{WRegion} parents and managers}

Each \type{WRegion} has a parent and possibly a manager associated to it.
The parent\index{parent} for an object is always a \type{WWindow} and for
\type{WRegion} with an X window (\type{WClientWin}, \type{WWindow}) the
parent \type{WWindow} is given by the same relation of the X windows.
For other \type{WRegion}s the relation is not as clear. There is generally
very few restrictions other than the above on the parent---child relation
but the most common is as described in Figure \ref{fig:parentship}.

\begin{figure}
\begin{verbatim}
    WScreens
     |
     |-->WViewports
     |
     |-->WIonWS:s and WFloatWS:s
     |
     |-->WClientWins in full screen mode
     |
     |-->WIonFrames and WFloatFrames
          |
          |-->WClientWins, including transients
          |
          |-->a possible WEdln or WMessage
\end{verbatim}
\caption{Most common parentship relations}
\label{fig:parentship}
\end{figure}

\type{WRegion}s have very little control over their children as a parent.
The manager\index{manager} \type{WRegion} has much more control over its
managed \type{WRegion}s. Managers, for example, handle resize requests,
focusing and displaying of the managed regions. Indeed the manager---managed
relationship gives a better picture of the logical ordering of objects on
the screen. Again, there are generally few limits, but the most common
hierarchy is given in Figure \ref{fig:managership}. Note that sometimes
the parent and manager are the same object and not all objects may have
a manager (e.g. the dock in the dock module at the time of writing this)
but all have a parent--a screen if not anything else.

\begin{figure}
\begin{verbatim}
    WScreens
     |
     |-->WViewports
          |
          |-->full screen WClientWins
          |    |
          |    |-->transient WClientWins (dialogs)
          |
          |-->WIonWSs
          |    |
          |    |-->WIonFrames
          |         |
          |         |-->WClientWins
          |         |    |
          |         |    |-->transient WClientWins (dialogs)
          |         |
          |         |-->a possible WEdln or WMessage
          |
          |-->WFloatWSs
               |
               |-->WFloatFrames
                    |
                    |-->WClientWins
                    |
                    |-->a possible WEdln or WMessage
\end{verbatim}
\caption{Most common manager--managed relations}
\label{fig:managership}
\end{figure}

Note how the \type{WClientWin}s managed by \type{WFloatFrame}s don't have
transients managed by them. This is because WFloatWSs choose to handle
transients differently (transients are put in separate frames like normal
windows; in the future they should be stacked above the frame containing the
\code{transient_for} window).
