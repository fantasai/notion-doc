\section{Winprops}
\label{sec:winprops}

The so-called ``winprops''\index{Winprops} can be used to change how
specific windows are handled and to set up some kludges to deal with
badly behaving applications. They are defined by calling the function
\code{defwinprop} with a table containing the properties to set and the
necessary information to identify a window. The currently supported
winprops are listed below, and the subsequent subsections explain the
usual method of identifying windows, and how to obtain this information.

%\begin{table}
%\begin{htmlonly}
%\docode % latex2html kludge
%\end{htmlonly}
%\caption{Supported winprops}
%\label{tab:winprops}

\newenvironment{winprop}[2]{
  \begin{function}%
      % Sigh. (La)TeX is a mess.
      %\index{%
      %  \ifx\\#1\\%
      %  #2\else#1\fi%
      %  @\expandafter\var{#2}}
      \item[Winprop:] \var{#1} (#2)
      \item[Description:]
}
{
  \end{function}
}


\begin{winprop}{acrobatic}{boolean}
    \index{acrobatic@\var{acrobatic}}
    Set this to \code{true} for Acrobat Reader. It has an annoying
    habit of trying to manage its dialogs instead of setting them as
    transients and letting the window manager do its job, causing
    Ion and acrobat go a window-switching loop when a dialog is
    opened. 
\end{winprop}


\begin{winprop}{float}{boolean}
    \index{float@\var{float}}
    Set this to open the window in a floating frame, when
    in a group.
\end{winprop}


\begin{winprop}{fullscreen}{boolean}
    \index{fullscreen@\var{fullscreen}}
    Should the window be initially in full screen mode?
\end{winprop}


\begin{winprop}{ignore_cfgrq}{boolean}
    \index{ignore-cfgrq@\var{ignore_cfgrq}}
    Should configure requests on the window be ignored?
    Only has effect on floating windows.
\end{winprop}


\begin{winprop}{ignore_net_active_window}{boolean}
    \index{ignore-net-active-window@\var{ignore_net_active_window}}
    Ignore extended WM hints \code{_NET_ACTIVE_WINDOW} request. 
\end{winprop}


\begin{winprop}{jumpto}{boolean}
    \index{jumpto@\var{jumpto}}
    Should a newly created client window always be made
    active, even if the allocated frame isn't. 
\end{winprop}


\begin{winprop}{new_group}{string}
    \index{new-group@\var{new_group}}
    If the region specified by \code{target} winprop does not exist
    (or that winprop is not set), create a new workspace using the 
    previously stored layout (see \fnref{ioncore.deflayout}) named by
    this property. After creating the workspace, \code{target} is 
    attempted to be found again. (If that still fails, the newly 
    created workspace is still asked to manage the client window.)
\end{winprop}


\begin{winprop}{oneshot}{boolean}
    \index{oneshot@\var{oneshot}}
    Discard this winprop after first use. 
\end{winprop}


\begin{winprop}{statusbar}{string}
    \index{statusbar@\var{statusbar}}
    Put the window in the statusbar, in the named tray component,
    (The default tray component is called simply \code{"systray"}, 
    and others you give names to in your custom template, always 
    prefixed by \code{"systray_"}.
\end{winprop}


\begin{winprop}{switchto}{boolean}
    \index{switchto@\var{switchto}}
    Should a newly mapped client window be switched to within
    its frame.
\end{winprop}
        

\begin{winprop}{target}{string}
    \index{target@\var{target}}
    The name of an object (workspace, frame) that should manage 
    windows of this type. See also \code{new_group}.
\end{winprop}

        
\begin{winprop}{transient_mode}{string}
    \index{transient-mode@\var{transient_mode}}
    \code{"normal"}: No change in behaviour. \code{"current"}:
    The window should be thought of as a transient for the current
    active client window (if any) even if it is not marked as a
    transient by the application. \code{"off"}: The window should 
    be handled as a normal window even if it is marked as a
    transient by the application. 
\end{winprop}


\begin{winprop}{transparent}{boolean}
    \index{transparent@\var{transparent}}
    Should frames be made transparent when this window is selected? \\
\end{winprop}


\subsection{Sizehint winprops}

Additionally, the winprops 
\code{max_size}\index{max-size@\var{max_size}},
\code{min_size}\index{min-size@\var{min_size}},
\code{aspect}\index{aspect@\var{aspect}},
\code{resizeinc}\index{aspect@\var{resizeinc}},
and
\code{ignore_max_size}\index{ignore-max-size@\var{ignore_max_size}},
\code{ignore_min_size}\index{ignore-min-size@\var{ignore_min_size}},
\code{ignore_aspect}\index{ignore-aspect@\var{ignore_aspect}},
\code{ignore_resizeinc}\index{ignore-aspect@\var{ignore_resizeinc}},
may be used to override application-supplied size hints. The four
first ones are tables with the fields \var{w} and \var{h}, indicating
the width and height size hints in pixels, and the latter ignore
winprop is a boolean. 

Finally, the boolean
\code{userpos}\index{userpos@\var{userpos}} option may be used to
override the \code{USPosition} flag of the size hints. Normally,
when this flag is set, Ion tries to respect the supplied window
position more than when it is not set. Obviously, this makes sense
only for floating windows.


\subsection{Classes, roles and instances}
\label{sec:classesrolesinstances}

The identification information supported include
\var{class}\index{class@\var{class}!winprop},
\var{role}\index{role@\var{role}!winprop},
\var{instance}\index{instance@\var{instance}!winprop},
\var{name}\index{name@\var{name}!winprop} and
\var{is_transient}\index{is-transient@\var{is_transient}!winprop}.
The first three of these are strings, and must exactly match the
corresponding information obtained from the window's properties.
The \var{name} field is a Lua-style regular expression matched against
the window's title. The \var{is_transient} field is a boolean that can
be used to include or exclude transients only. It is not necessary 
to specify all of these fields.

Ion looks for a matching winprop in the order listed by the following
table. An 'E' indicates that the field must be set in the winprop
and it must match the window's corresponding property exactly or, in
case of \var{name}, the regular expression must match the window
title. An asterisk '*' indicates that a winprop where the field is
not specified (or is itself an asterisk in case of the first three
fields) is tried.

\begin{center}
\begin{tabular}{llll}
  \tabhead{\var{class} & \var{role} & \var{instance} & other}
  E	       & E          & E              & E \\
  E	       & E          & E              & * \\
  E	       & E          & *              & E \\
  E	       & E          & *              & * \\
  E	       & *          & E              & E \\
  E	       & *          & E              & * \\
  E	       & *          & *              & E \\
  \vdots       & \vdots     & \vdots         & etc. \\
\end{tabular}
\end{center}

If there are multiple matching winprops with the same
\var{class}, \var{role} and \var{instance}, but other information
different, the most recently defined one is used.


\subsection{Finding window identification}

The 'Window info' context menu entry (\key{Mod1+M} or \key{Button3} on a tab)
can be used to list the identification information required to set winprops
for a window and all the transient windows managed within it. 

\index{xprop} 
Another way to get the identification information is to use \command{xprop}.
Simply run To get class and instance, simply run \command{xprop WM_CLASS}
and click on the particular window of interest. The class is the latter of
the strings while the instance is the former.  To get the role -- few
windows have this property -- use the command \command{xprop WM_ROLE}. 
This method, however, will not work on transients. 

\index{transient}
So-called ``transient windows'' are usually short-lived dialogs (although
some programs abuse this property) that have a parent window that they are
``transient for''. On tiled workspaces Ion displays these windows 
simultaneously with the parent window at the bottom of the same frame.
Unfortunately \command{xprop} is stupid and can't cope with this situation,
returning the parent window's properties when the transient is clicked on.
For this reason you'll have to do a little extra work to get the properties
for that window.\footnote{There's a patch to \command{xprop} to
fix this, but nothing seems to be happening with respect to including it in 
XFree86.}

Finally, it should be mentioned that too many authors these days
``forget'' to set this vital identification to anything meaningful:
everything except name is the same for all of the program's 
windows, for example. Some other programs only set this information
after the window has been mapped, i.e. the window manager has been
told to start managing it, which is obviously too late. 
Gtk applications in particular are often guilty on both counts.


\subsection{Some common examples}

\subsubsection{Acrobat Reader}

The following is absolutely necessary for Acrobat reader:

\begin{verbatim}
defwinprop{
    class = "AcroRead",
    instance = "documentShell",
    acrobatic = true,
}
\end{verbatim}

\subsubsection{Fixing a Mozilla Firebird transient}

Mozilla Firebird (0.7) incorrectly does not set the \code{WM_TRANSIENT_FOR} 
property for the dialog that is used to ask the action to take for a file.
It, however, sets the the property point to the main window for the save
dialog. This can be annoying and confusing, as the first dialog is not 
closed before the second is displayed.

We'd like the first dialog to be transient to the main window. The closest
we can get to that is to consider it transient to the current window (if
there's one). Unfortunately Firebird does not set any meaningful classes, 
instances or roles for the windows, so we'll have to rely on an ugly title
match.

\begin{verbatim}
defwinprop{
    class = "MozillaFirebird-bin",
    name = "Opening .*",
    transient_mode = "current",
}
\end{verbatim}

\subsubsection{Forcing newly created windows in named frames}

The following winprop should place xterm started with command-line parameter
\mbox{\code{-name sysmon}} and running a system monitoring program in a
particular frame:
\begin{verbatim}
defwinprop{
    class = "XTerm",
    instance = "sysmon",
    target = "sysmonframe",
}
\end{verbatim}

For this example to work, we have to somehow create a frame named
\code{sysmonframe}. One way to do this is to make the following
call in the \key{Mod1+F3} Lua code query:

\begin{verbatim}
mod_query.query_renameframe(_)
\end{verbatim}

Recall that \code{_} points to the multiplexer (frame or screen) in which 
the query was opened. Running this code should open a new query prefilled
with the current name of the frame. In our example we would change the 
name to \code{sysmonframe}, but we could just as well have used the 
default name formed from the frame's class name and an instance number.
