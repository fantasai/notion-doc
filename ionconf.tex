\documentclass[english,a4paper,11pt,oldtoc,mctitle]{rapport3}
\usepackage{babel}
\usepackage[latin1]{inputenc}
\usepackage[T1]{fontenc}
\usepackage{palatino}
%\usepackage{ae}
\usepackage{url}
%\usepackage{graphicx}
%\usepackage{color}
\usepackage{makeidx}
\usepackage{tabularx}
%\usepackage{textcomp}
\usepackage[nottoc]{tocbibind}
\usepackage{enumerate} % GNU FDL needs this
\usepackage{calc}
\usepackage{ifpdf}
\usepackage[a4paper,margin=3cm]{geometry}

\ifpdf
\usepackage[pdftex]{hyperref}
\else
\usepackage[hypertex]{hyperref}
\fi

\usepackage{html}

% URL settings
%%%%%%%%%%%%%%%%%%%%%%%%%%%%%%%%%%%%%%%%%%%%%%%%%%%%%%

\urlstyle{tt}

% listings package
%%%%%%%%%%%%%%%%%%%%%%%%%%%%%%%%%%%%%%%%%%%%%%%%%%%%%%

%begin{latexonly}
\usepackage{listings}

\lstset{
  extendedchars=true,
  breaklines=true,
  basicstyle=\tt,
  alsoletter={",},
  alsoother={\_},
}
%end{latexonly}

\sloppy

% Some markup
%%%%%%%%%%%%%%%%%%%%%%%%%%%%%%%%%%%%%%%%%%%%%%%%%%%%%%

\newcommand{\note}[1]{\color{red}**#1**}
\newcommand{\type}[1]{#1}
\newcommand{\spec}[1]{#1}
\newcommand{\indextype}[1]{\index{#1@\type{#1}}}

%begin{latexonly}
\newcommand{\file}[1]{\mbox{\emph{#1}}}
\newcommand{\key}[1]{\mbox{\textbf{#1}}}
\newcommand{\code}[1]{\lstinline!#1!}
\newcommand{\codestr}[1]{`\texttt{#1}'}
\newcommand{\var}[1]{\lstinline!#1!}
\newcommand{\command}[1]{\lstinline!#1!}

\newcommand{\hyperlabel}[1]{\hypertarget{#1}{}\label{#1}}
\newcommand{\fnrefx}[2]{\hyperlink{fn:#1.#2}{\code{#2}}}
\newcommand{\fnref}[1]{\hyperlink{fn:#1}{\code{#1}}}
\newcommand{\myhref}[2]{\hyperlink{#1}{#2}}

%end{latexonly}
\begin{htmlonly}
    
\newcommand{\file}[1]{\emph{#1}}
\newcommand{\key}[1]{\textbf{#1}}
\newcommand{\code}[1]{\texttt{#1}}
\newcommand{\codestr}[1]{`\texttt{#1}'}
\newcommand{\var}[1]{\texttt{#1}}
\newcommand{\command}[1]{\texttt{#1}}

\newcommand{\hyperlabel}[1]{\label{#1}}
\newcommand{\fnref}[1]{\htmlref{\texttt{#1}}{fn:#1}}
\newcommand{\fnrefx}[2]{\htmlref{\texttt{#2}}{fn:#1.#2}}
\newcommand{\myhref}[2]{\htmlref{#2}{#1}}

\end{htmlonly}

\newcommand{\tabhead}[1]{\hline #1 \\ \hline}

% function tables
%%%%%%%%%%%%%%%%%%%%%%%%%%%%%%%%%%%%%%%%%%%%%%%%%%%%%%

%begin{latexonly}

\newcommand{\funclistlabel}[1]{#1\hfill}
\newenvironment{function}{
  \pagebreak[3]
  \begin{list}{}{
        \settowidth{\labelwidth}{Description:}
        \setlength{\leftmargin}{\labelwidth}
        \addtolength{\leftmargin}{0.5em}
        \setlength{\labelsep}{0.5em}
        \setlength{\itemsep}{0pt}
        \setlength\parsep{0pt}
        \setlength\topsep{0pt}
        %\setlength{\itemsep}{-\parskip}
        %\addtolength{\itemsep}{\lineskip}
        \let\makelabel\funclistlabel
  }
}{
  \end{list}
}

%end{latexonly}
\begin{htmlonly}

\newenvironment{function}{
  \begin{description}
}{
  \end{description}
}

\end{htmlonly}


\newcommand{\synopsis}[1]{
  \item[Synopsis:] \code{#1}
}
\newcommand{\funcname}[1]{
  \item[Function:] \code{#1}
}
\newcommand{\hookname}[1]{
  \item[Hook name:] \hyperlabel{#1}\code{#1}
}
\newcommand{\hookparams}[1]{
  \item[Parameters:] \code{#1}
}
\newenvironment{funcdesc}{
  \item[Description:]
}{}


% While rapport3/artikel3 are otherwise nice classes, 
% itemize looks awful.
%%%%%%%%%%%%%%%%%%%%%%%%%%%%%%%%%%%%%%%%%%%%%%%%%%%%%%
%begin{latexonly}
\makeatletter
\renewenvironment{itemize}{%
  \ifnum \@itemdepth >3
    \@toodeep
  \else
    \advance\@itemdepth \@ne
    \edef\@itemitem{labelitem\romannumeral\the\@itemdepth}%
    \list{\csname\@itemitem\endcsname}%
         {%
            \if@revlabel
              \def\makelabel##1{\hskip .5\unitindent{\hfil ##1}}\else
	      					    %^^^^^^^^^ Changed
              \def\makelabel##1{\hfil ##1}
	      		       %^^^^^^^^^ Changed
            \fi
          }%
  \fi}
 {\global\@ignoretrue \endlist}
\makeatletter
%end{latexonly}


% For including some files from articles
\newcommand{\xchapter}[1]{\chapter{#1}}
\newcommand{\xsection}[1]{\section{#1}}
\newcommand{\xsectionstar}[1]{\section*{#1}}
\newcommand{\xsubsection}[1]{\subsection{#1}}


\title{Ion: Configuring and extending with Lua}
%begin{latexonly}
\author{Tuomo Valkonen \\ tuomov@iki.fi}
%end{latexonly}
\begin{htmlonly}
\author{Tuomo Valkonen \\ tuomov at iki.fi}
\end{htmlonly}
%%DATE

\makeindex


\begin{document}

\maketitle

Copyright \copyright\  2003 Tuomo Valkonen.
Permission is granted to copy, distribute and/or modify this document
under the terms of the GNU Free Documentation License, Version 1.2
or any later version published by the Free Software Foundation;
with no Invariant Sections, no Front-Cover Texts, and no Back-Cover Texts.
A copy of the license is included in the section entitled ''GNU
Free Documentation License''.

\begin{abstract}
    This document is a sort of ''advanced user'' manual for Ion. It
    attempts at documenting what is in Ion's configuration files,
    how to configure Ion by simple modifications to these files and
    how to write more complex extensions in Lua. As an essential 
    background the document first, however, explains Ion's object and
    class hierarchies. Finally, the document contains the essential
    reference on most of the functions exported by the C side of Ion
    for use in Lua code.
\end{abstract}

\tableofcontents


\section{Class and object hierarchies}
\label{sec:objects}

While Ion does not have a truly object-oriented design
\footnote{the author doesn't like such artificial designs},
things that appear on the computer screen are, however, quite
naturally expressed as such ``objects''. Therefore Ion implements
a rather primitive OO system for these screen objects and some
other things. 

It is essential for the module writer to learn this object
system, but also people who write their own binding configuration files
necessarily come into contact with the class and object hierarchies
-- you need to know which binding setup routines apply where, 
and what functions can be used as handlers in which bindings.
It is the purpose of this section to attempt to explain these 
hierarchies. If you do not wish the read the full section, at least
read the summary at the end of it, so that you understand the very
basic relations.

For simplicity we consider only the essential-for-basic-configuration
Ioncore, \file{mod\_tiling} and \file{mod\_query} classes. 
See Appendix \ref{app:fullhierarchy} for the full class hierarchy visible
to Lua side.

\subsection{Class hierarchy}

One of the most important principles of object-oriented design methodology
is inheritance; roughly how classes (objects are instances of classes)
extend on others' features. Inheritance gives rise to class hierarchy.
In the case of single-inheritance this hierarchy can be expressed as a
tree where the class at the root is inherited by all others below it
and so on. Figure \ref{fig:classhierarchy} lists out the Ion class 
hierarchy and below we explain what features of Ion the classes 
implement.

\begin{figure}
\begin{htmlonly}
\docode % latex2html kludge
\end{htmlonly}
\begin{verbatim}
    Obj
     |-->WRegion
     |    |-->WClientWin
     |    |-->WWindow
     |    |    |-->WMPlex
     |    |    |    |-->WFrame
     |    |    |    `-->WScreen
     |    |    |         `-->WRootWin
     |    |    `-->WInput (mod_query)
     |    |         |-->WEdln (mod_query)
     |    |         `-->WMessage (mod_query)
     |    |-->WGroup
     |    |    |-->WGroupWS
     |    |    `-->WGroupCW
     |    `-->WTiling (mod_tiling)
     `-->WSplit (mod_tiling)
\end{verbatim}
\caption{Partial Ioncore, \file{mod\_tiling} and \file{mod\_query} 
    class hierarchy.}
\label{fig:classhierarchy}
\end{figure}

The core classes:

\begin{description}
  \item[\type{Obj}]\indextype{Obj}
    Is the base of Ion's object system.

  \item[\type{WRegion}]\indextype{WRegion}
    is the base class for everything corresponding to something on the
    screen. Each object of type \type{WRegion} has a size and  position
    relative to the parent \type{WRegion}. While a big part of Ion 
    operates on these instead of more specialised classes, \type{WRegion}
    is a ``virtual''  base class in that there are no objects of ``pure''
    type \type{WRegion}; all concrete regions are objects of some class 
    that inherits \type{WRegion}.

  \item[\type{WClientWin}]\indextype{WClientWin} is a class for
    client window objects, the objects that window managers are
    supposed to manage.

  \item[\type{WWindow}]\indextype{WWindow} is the base class for all
    internal objects having an X window associated to them
    (\type{WClientWins} also have X windows associated to them).
    
  \item[\type{WMPlex}] is a base class for all regions that ``multiplex'' 
    other regions. This means that of the regions managed by the multiplexer,
    only one can be displayed at a time. 
  
  \item[\type{WScreen}]\indextype{WScreen} is an instance of \type{WMPlex}
    for screens.
    
  \item[\type{WRootWin}]\indextype{WRootWin} is the class for
    root windows\index{root window} of X screens\index{screen!X}.
    It is an instance of \type{WScreen}.
    Note that an ``X screen'' or root window is not necessarily a
    single physical screen\index{screen!physical} as a root window
    may be split over multiple screens when ugly hacks such as 
    Xinerama\index{Xinerama} are used. (Actually there can be only 
    one root window when Xinerama is used.) 
    
  \item[\type{WFrame}]\indextype{WFrame} is the class for frames.
    While most Ion's objects have no graphical presentation, frames 
    basically add to \type{WMPlex}es the decorations around client 
    windows (borders, tabs).
    
  \item[\type{WGroup}]\indextype{WGroup} is the base class for groups.
    Particular types of groups are workspaces 
    (\type{WGroupWS}\indextype{WGroupWS})
    and groups of client windows
    (\type{WGroupCW}\indextype{WGroupCW}).
\end{description}


Classes implemented by the \file{mod\_tiling} module:

\begin{description}
  \item[\type{WTiling}]\indextype{WTiling} is the class for tilings
    of frames.
  \item[\type{WSplit}]\indextype{WSplit} (or, more specifically, classes
    that inherit it) encode the \type{WTiling} tree structure.
\end{description}


Classes implemented by the \file{mod\_query} module:

\begin{description}
  \item[\type{WInput}]\indextype{WInput} is a virtual base class for the
    two classes below.
  \item[\type{WEdln}]\indextype{WEdln} is the class for the ``queries'',
    the text inputs that usually appear at bottoms of frames and sometimes
    screens. Queries are the functional equivalent of ``mini buffers'' in
    many text editors.
  \item[\type{WMessage}]\indextype{WMessage} implements the boxes for 
    warning and other messages that Ion may wish to display to the user. 
    These also usually appear at bottoms of frames.
\end{description}

There are also some other ``proxy'' classes that do not refer
to objects on the screen. The only important one of these for
basic configuration is \type{WMoveresMode} that is used for
binding callbacks in the move and resize mode.

\subsubsection{Run-time access to types}

Even though it often indicates a design mistake, sometimes it can be useful to
have run-time access to the types of objects.

For example, to check wether a given object is of type WMPlex, the following C
construction can be used:

\begin{verbatim}
    if(obj_is((Obj*)cwin, &CLASSDESCR(WMPlex)))
        ....
\end{verbatim}

Its lua counterpart is:

\begin{verbatim}
    if(obj_is(cwin, "WMPlex"))
        ....
\end{verbatim}

While there's also an 'obj\_cast' method available, C structs can be freely 
cast to their 'superclass' using a regular C cast, for example:

\begin{verbatim}
bool input_fitrep(WInput *input, WWindow *par, const WFitParams *fp)
{
    if(par!=NULL && !region_same_rootwin((WRegion*)input, (WRegion*)par))
        return FALSE;
    ...
\end{verbatim}

\subsection{Object hierarchies: \type{WRegion} parents and managers}

\subsubsection{Parent--child relations}
Each object of type \type{WRegion} has a parent and possibly a manager
associated to it. The parent\index{parent} for an object is always a 
\type{WWindow} and for \type{WRegion} with an X window (\type{WClientWin},
\type{WWindow}) the parent \type{WWindow} is given by the same relation of
the X windows. For other \type{WRegion}s the relation is not as clear.
There is generally very few restrictions other than the above on the
parent---child relation but the most common is as described in
Figure \ref{fig:parentship}.

\begin{figure}
\begin{htmlonly}
\docode % latex2html kludge
\end{htmlonly}
\begin{verbatim}
    WRootWins
     |-->WGroupWSs
     |-->WTilings
     |-->WClientWins in full screen mode
     `-->WFrames
          |-->WGroupCWs
          |-->WClientWins
          |-->WFrames for transients
          `-->a possible WEdln or WMessage
\end{verbatim}
\caption{Most common parent--child relations}
\label{fig:parentship}
\end{figure}

\type{WRegion}s have very little control over their children as a parent.
The manager\index{manager} \type{WRegion} has much more control over its
managed \type{WRegion}s. Managers, for example, handle resize requests,
focusing and displaying of the managed regions. Indeed the manager---managed
relationship gives a better picture of the logical ordering of objects on
the screen. Again, there are generally few limits, but the most common
hierarchy is given in Figure \ref{fig:managership}. Note that sometimes
the parent and manager are the same object and not all regions may have
a manager, but all non-screen regions have a parent---a screen if not 
anything else.

\subsubsection{Manager--managed relations}

\begin{figure}
\begin{htmlonly}
\docode % latex2html kludge
\end{htmlonly}
\begin{verbatim}
    WRootWins
     |-->WGroupCWs for full screen WClientWins
     |    |-->WClientWins
     |    `-->WFrames for transients (dialogs)
     |         `--> WClientWin
     |-->WGroupWSs for workspaces
     |    |-->WTiling
     |    |    |-->WFrames
     |    |    |    `-->WGroupCWs (with contents as above)
     |    |    `-->possibly a WStatusBar or WDock
     |    |-->WFrames for floating content
     |    |-->possibly a WEdln, WMessage or WMenu
     |    `-->possibly a WStatusBar or WDock (if no tiling)
     `-->WFrames for sticky stuff, such as the scratchpad
\end{verbatim}
\caption{Most common manager--managed relations}
\label{fig:managership}
\end{figure}

Note that a workspace can manage another workspace. This can be
achieved with the \fnref{attach_new} function, and allows you to nest
workspaces as deep as you want.

%Note how the \type{WClientWin}s managed by \type{WFloatFrame}s don't have
%transients managed by them. This is because WFloatWSs choose to handle
%transients differently (transients are put in separate frames like normal
%windows).

\subsection{Summary}

In the standard setup, keeping queries, messages and menus out of
consideration:

\begin{itemize}
  \item The top-level objects that matter are screens and they correspond
    to physical screens. The class for screens is \type{WScreen}.
  \item Screens contain (multiplex) groups (\type{WGroup}) and other 
    objects, such as \type{WFrames}. Some of these are mutually exclusive
    to be viewed at a time.
  \item Groups of the specific kind \type{WGroupWS} often contain a
    \type{WTiling} tiling for tiling frames (\type{WFrame}), but 
    groups may also directly contain floating frames.
  \item Frames are the objects with decorations such as tabs and borders.
    Frames contain (multiplex) among others (groups of) client windows, 
    to each of which corresponds a tab in the frame's decoration. Only 
    one client window (or other object) can be shown at a time in each 
    frame. The class for client windows is \type{WClientWin}.
\end{itemize}




\chapter{Basic configuration}
\label{chap:config}

This chapter should help the reader configure Ion to her/his liking. As 
the reader probably already knows, Ion uses Lua as a configuration and 
extension language. If the reader is new to Lua, he might first want
to read some Lua documentation as already suggested and pointed to in the
Introduction before continuing with this chapter.

In particular, if ''anonymous function'' sounds or the function construct in
\begin{verbatim}
kpress("Mod1+1", function(s) screen_switch_nth(s, 0) end)
\end{verbatim}
looks confusing to you, please consider reading some Lua documentation,
\url{http://lua-users.org/wiki/FunctionsTutorial} if nothing else.
Ion's stock configuration files use anonymous functions quite extensively.

Section \ref{sec:conffiles}�is an overview of the multiple configuration
files Ion uses and as a perhaps more understandable introduction to the
general layout of the configuration files, a walk-through of the main 
configuration file \file{ion.lua} is provided in section 
\ref{sec:walkthrough}.
How keys and mouse action are bound to functions is described in detail
in \ref{sec:bindings} and in section \ref{sec:winprops} winprops are
explained. For a reference on exported functions, see section
\ref{sec:exports}.

\section{The configuration files}
\label{sec:conffiles}

The development branch of Ion (to which document applies) stores its stock
configuration files in \file{/usr/local/etc/ion/} unless 
you, OS package maintainer or whoever installed the package has modified 
the variables
\code{PREFIX}\index{PREFIX@\code{PREFIX}} or
\code{ETCDIR}\index{ETCDIR@\code{ETCDIR}} in
\file{system.mk}\index{system.mk@\file{system.mk}} before compiling Ion.
In the first case you probably know where to find the files and in 
the other case the system administrator or OS package maintainer should 
have provided documentation to point to the correct location. If these
instructions are no help in locating the correct directory, the command
\code{locate ion.lua} might help provided \code{updatedb} has been
run recently. User configuration files go in \file{\~{}/.ion2/}. 

Ion always searches user configuration file directory before the stock
configuration file directory for files. Therefore, if you want to change
some setting, it is advised against that you modify the stock configuration
files in-place as subsequent installs of Ion will restore the stock
configuration files. Instead you should always make a copy of the stock
file in \file{\~{}/.ion2/} and modify this file. When searching
for a file, if no extension or path component is given, compiled \file{.lc} 
files are attempted before \file{.lua} files.

The ''Ioncore'' window manager core and each module have their own
configuration files that should be used to configure that module. 
The configuration files related to the ioncore main binary are as
follows. The files \file{ion.lua} and \file{draw.lua} are loaded 
from Ioncore and the rest are included from the former.

\begin{tabularx}{\linewidth}{
      p{\widthof{ion-bindings.lua}}%
      X}
    \hline
    File & Description \\
    \hline
    \file{ion.lua} & 
    The main configuration file \\
    %
    \file{ion-bindings.lua} & 
    Most of Ion's bindings are configured here. Bindings that are
    specific to some module are configured in the module's configuration
    file. For details, see section \ref{sec:bindings}. \\
    \file{ion-menus.lua} & 
    Menu definitions; see section \ref{sec:menus}. \\
    %
    \file{kludges.lua} & 
    Settings to get some apps behave more nicely have been collected here.
    See section \ref{sec:winprops} for details on these ''winprops''. \\
    %
    \file{draw.lua} &
    This file is a link to or copy of one of the \file{look-*.lua} style
    files. It should load a drawing engine and configure a style for
    it; for details see chapter \ref{chap:gr}. \\
\end{tabularx}

Each (non-drawing engine) module has in addition its own configuration
file loaded when that module is loaded:

\begin{tabularx}{\linewidth}{
      p{\widthof{ion-bindings.lua}}%
      X}
    \hline
    File & Description \\
    \hline
    \file{ionws.lua} & 
    Configuration file for the ionws module. Bindings specific to the
    workspace and frame classes implemented by this module are
    configured here. \\
    %
    \file{floatws.lua} & 
    Configuration file for the floatws module. Bindings specific to
    the workspace and frame classes implemented by this module are
    configured here. \\
    %
    \file{query.lua} & 
    Configuration file for the query module. Bindings to edit text
    in the queries and some other bindings related to queries and
    messages are defined here. \\
    %
    \file{menu.lua} & 
    Configuration file for the menu module. Bindings to navigate
    menus are defined here. Actual menus are (in the stock
    configuration file setup) defined in \file{ion-menus.lua}
    as mentioned above. \\
\end{tabularx}

Some of the files contain references to the files \file{querylib.lua}
and \file{menulib.lua}
These are installed in
\code{SHAREDIR}\index{SHAREDIR@\code{SHAREDIR}}
(\file{/usr/local/share/ion/} by default)
among some other other \file{.lua} files that are an essential part of
Ion's code. Users who only want to change a few settings should not
need to modify the files in this directory. Nevertheless, it is
possible to override the files in \code{SHAREDIR} as it is on the search
path after \code{\~/.ion2} and \code{ETCDIR}.

There is one extra file in \code{SHAREDIR} that you may find usefull and
that is not loaded by default. This is \file{compat.lua} and it contains
some wrapper functions for backwards compatibility to make the process of
updating Ion a little less painfull. If you have have modified configuration
files that use some features no longer available in the latest Ion, you may
just load this file (with \code{include("compat.lua")} at the beginning of
\file{ion.lua}, for example) instead of immediately updating your
configuration files. However, the wrappers will be removed eventually
(maybe about two months after adding them depending on the rate of new
releases) so you should nevertheless update your configuration files before
this happens.

\section{A walk through \file{ion.lua}}
\label{sec:walkthrough}

As already mentioned \file{ion.lua} is Ion's main configuration
file. Some basic 'feel' settings are usually configured there and
the necessary modules and other configuration files configuring some 
more specific aspects of Ion are loaded there. In this section we
take a walk through the stock \file{ion.lua}.

The first thing that is done in that file is set
\begin{verbatim}
DEFAULT_MOD = "Mod1+"
\end{verbatim}
This causes most of Ion's key bindings to use \key{Mod1} as the
modifier key; for details on modifiers and key binding setup in 
general see section \ref{sec:bindings}.

Next there are the commented-out delay settings
\begin{verbatim}
-- set_dblclick_delay(250)
-- set_resize_delay(1500)
\end{verbatim}
The first of these settings is the maximum interlval in milliseconds 
between two mouse button presses for the second press to be actually
considered a double-click. The latter setting sets the delay, again 
in milliseconds, after which Ion will automatically terminate 
one of the keyboard resize modes if nothing has happened. 

The setting
\begin{verbatim}
enable_opaque_resize(false)
\end{verbatim}
says that a XOR rubberband should be shown when moving or resizing
frames. This will, unfortunately, cause Ion to also grab the X
server and has some side effects. If your computer is fast enough
and you prefer so-called ''opaque resize'' mode where the frame is
being resized already during the resize action  you may set the parameter
to \verb!true!.

The following settings controls whether Ion will ''warp'' the (mouse)
pointer to an object whenever it is focused from the keyboard. 
\begin{verbatim}
enable_warp(true)
\end{verbatim}
Some people may consider this annoying so setting the parameter
to \verb!false! can be used to disable this feature and have Ion
never move the pointer.

Next the stock \file{ion.lua} has the include-statements
\begin{verbatim}
include("kludges")
include("ion-bindings")
include("ion-menus")
\end{verbatim}
The first of these loads the file \file{kludges.lua} that contains
some ''winprop'' settings used to make some applications behave a
little better under Ion. For details see section \ref{sec:winprops}.
The second include statement loads file file 
\file{ion-bindings.lua} which containts the statements to configure 
most of Ion's bindings; modules' bindings are configured in their 
respective configuration files. The final include statement load
menu definitions.

Next we have quite a few statements of the form
\begin{verbatim}
add_shortenrule("[^:]+: (.*)(<[0-9]+>)", "$1$2$|$1$<...$2")
\end{verbatim}
These are used to configure how Ion attempts to shorten window titles
when they do not fit in a Tab. The first argument is a POSIX regular
expression that is used to match against the title and the next is
a rule to construct a new title of a match occurs. This particular
rule is used to shorten e.g. 'Foo: barbaz<3>' to 'barba{\ldots}<3>'; for
details see the function reference entry for \fnref{add_shortenrule}.

The setting
\begin{verbatim}
query_man_path = {
    "/usr/man",
    "/usr/share/man",
    "/usr/X11R6/man",
    "/usr/local/man"
}
\end{verbatim}
is used to configure where the \fnref{querylib.query_man} query
(\key{F1}) looks for man pages to tab-complete.

Finally we load the modules.
\begin{verbatim}
load_module("query")
load_module("menu")
load_module("ionws")
load_module("floatws")
\end{verbatim}
The first of the modules provides the queries, the minibuffer-like
line-editor boxes that appear at bottoms of frames and screens and that
can be and are used to to do quite a few things including starting
programs and navigating to different windows by their name.
The 'menu' module provides both pop-up and query-style in-frame menus.
The third module loaded provides the normal Ion-style tiled workspaces
and frames of that style while the 'floatws' module provides traditional
''free-floating'' WIMP workspaces and frames in the PWM style.

As already mentioned, each of these modules have their own configuration
files \file{modulename.lua} that configure things that can only be
configured if the module is loaded. These settings are mostly just the
bindings that are specific to the classes provided by the module.

At the moment there are no other modules than the above three. Any of
these modules may be removed; Ion is able to operate without any modules
in full-screen  only mode, but if you do so, you might first want to 
reconfigure some of the bindings. Ion will, however, complain of broken
workspace savefiles if starting a module-less configuration over an old
session.

\section{Keys and rodents}
\label{sec:bindings}

In the stock configuration file setup, most key and mouse bindings are set
from the file \file{ion-bindings.lua} while module-specific bindings
are set from the modules' main configuration files (\file{modulename.lua}).
This, however, does not have to be so as long as the module has been
loaded prior to defining any module-specific bindings.
%Ion's modules no longer fail to load if the bindings have not been set
%during the execution of the module's configuration file, so the bindings
%could now be set from elsewhere. In any case, the module must have been
%loaded prior to defining any bindings.

The bindings are defined by calling various functions that add a new set
of bindings to a binding group, usually related to some object class.
These functions, descriptions of their contexts and parameters passed to 
''bindings handlers'' are listed in the following subsection. 
Each of the functions listed there has a single argument: a table listing
the keys and mouse actions to bind functions to. Section
\ref{sec:binddef} describes how the binding tables can be constructed.

Note that when multiple objects want to handle a binding, the innermost
(when the root window is considered the outermost) active object in the
parent--child hierarchy (see Figure \ref{fig:parentship}) of objects 
gets to handle the action.

\subsection{The binding setup functions}

This section simply lists out the different binding setup functions
Ioncore and the modules provide and does not even attempt to describe
how to use these functions; for that refer to \ref{sec:binddef} and
following sections.

There has been some confusion among users about the need for multiple, 
partially even overlapping functions to setup bindings so let me try
give reasons for this design decision here. 

The thing is that if there was a just a single 'bind this key
to this function' method to bind keys, some limitations would have to
be made on the available functions or writing custom functions would
be more complicated. In addition one may want to bind the same function
to different key for different types of object. Indeed, the workspace
and frame tab swithing functions are the same both classes being based
on \type{WMPlex}, and in the stock configuration the switch to $n$:th 
workspaces is bound to \key{Mod1+n} (using \fnref{global_bindings}
explained below) while the switch to $n$:th tab is bound to the
sequence \key{Mod1+k n}  (using \fnref{genframe_bindings} and 
\fnref{submap}). Of course all this inheritance could be hidden
from the configuration interface by duplicating functions, but then
at what level this should be then done? There would quickly be a lot
of essentially same functions (\code{screen_switch_nth},
\code{genframe_switch_nth}, \code{ionframe_switch_nth},
\code{floatframe_switch_nth}, and so on for every new type of 
multiplexer added) and that prospect does not look so good.

So, in brief this little additional complexity in configuration is the
price for flexibility and in part elegance (if there can be said to be
anything of the like in Ion).

\subsubsection{Ioncore binding setup functions}

The following binding setup functions are defined by Ioncore and in the
stock configuration file setup, they are set from the file 
\file{ion-bindings.lua}:

\begin{tabularx}{\linewidth}{lX}
    \hline
    Function & Handler parameters and description \\
    \hline
    \fnref{global_bindings} & Parameters to handler: \type{WScreen} \\
			    & Description: Bindings that are available all the
			      time. \\
    \fnref{mplex_bindings}  & Parameters to handler: \type{WMPlex} \\
			    & Description: Bindings that are common to all
			      \type{WMPlex}es (screens and frames). Usually
			      only bindings that affect current client window
			      (if any) are set here. \\
    \fnref{genframe_bindings} & Parameters to handler: 
    			      \type{WGenFrame}, [\type{WRegion}] \\
			    & Description: Bindings that are common to all
			      types of frames (but not screens unlike
			      \code{mplex_bindings} above. When a tab has been
			      pressed the \type{WRegion} corresponding to the
			      tab is passed as the extra parameter. \\
\end{tabularx}


\subsubsection{IonWS module binding setup functions}

The following binding setup functions are defined by the IonWS module
and in the stock configuration file setup the bindings are set in
\file{ionws.lua}:

\begin{tabularx}{\linewidth}{lX}
    \hline
    Function & Handler parameters and description \\
    \hline
    \fnref{ionws_bindings}  & Parameters to handler: 
                              \type{WIonWS}, \type{WRegion} \\
			    & Description: Bindings that are available on the
			      tiled workspaces implemented by this module.
			      The extra parameter to binding handler is the 
			      currently active object on the workspace. \\
    \fnref{ionframe_bindings} & Parameters to handler: 
    			      \type{WIonFrame}, [\type{WRegion}] \\
			    & Description: Bindings that are specific to 
			      the tiled frames. As above, when a tab has been
			      pressed the \type{WRegion} corresponding to the
			      tab is passed as the extra parameter. \\
    \fnref{ionframe_moveres_bindings} & Parameters to handler: 
                              \type{WIonFrame} \\
			    & Description: Resize mode bindings.  Activated 
			      by calling \fnref{WIonFrame.begin_resize}. Only
			      certain functions may be called here; see the
			      function reference for details. \\
\end{tabularx}

\subsubsection{FloatWS module binding setup functions}

These functions are similar to the ones described in the above section for 
the FloatWS module. The bindings are defined in the configuration file 
\file{floatws.lua}:

\begin{tabularx}{\linewidth}{lX}
    \hline
    Function & Handler parameters and description \\
    \hline
    \fnref{floatws_bindings}& Parameters to handler: 
    			      \type{WFloatWS}, [\type{WRegion}] \\
			    & Description: Bindings that are available on the
			      conventional workspaces implemented by this 
			      module. The extra parameter to binding handler
			      is the  currently active object on the 
			      workspace, if any. \\
    \fnref{floatframe_bindings} & Parameters to handler: 
    			      \type{WFloatFrame}, [\type{WRegion}] \\
			    & Description: Bindings that are specific to 
			      the the conventional floating frames. \\
    \fnref{floatframe_moveres_bindings} & Parameters to handler:
                              \type{WIonFrame} \\
			    & Description: Keyboard move/resize mode 
			      bindings. Activated by calling the function
			      \fnref{WFloatFrame.begin_resize}. Only certain
			      functions may be called here; see the
			      function reference for details. \\
\end{tabularx}


\subsubsection{Query module binding setup functions}

These functions set the bindings for the query module. The bindings are set 
from the file \file{query.lua}.

\begin{tabularx}{\linewidth}{lX}
    \hline
    Function & Handler parameters and description \\
    \hline
    \fnref{input_bindings}  & Parameters to handler: \type{WInput} \\
			    & Description: bindings that are common to
			      message and query boxes; stuff to close
			      the box and to scroll message or completions. \\
    \fnref{query_bindings}  & Parameters to handler: \type{WEdln} \\
			    & Description: Bindings to edit text and
			      finish the query.\\
\end{tabularx}


\subsubsection{Menu module binding setup functions}

These functions set the bindings for the menu module. The bindings are set 
from the file \file{menu.lua}.

\begin{tabularx}{\linewidth}{lX}
    \hline
    Function & Handler parameters and description \\
    \hline
    \fnref{menu_bindings}  & Parameters to handler: \type{WMenu} \\
			   & Description: key bindings available in in-frame 
			     menus. \\
\end{tabularx}

\subsection{Defining the bindings}
\label{sec:binddef}

Each of the functions listed above has a single argument: a table listing
the key presses and other actions to be bound to functions. The descriptions
of individual bindings in this table are also tables that can be more 
conveniently constructed with the following functions:

Key presses:
\begin{itemize}
    \item \fnref{kpress}\code{(keyspec, func)},
    \item \fnref{kpress_waitrel}\code{(keyspec, func)} and
    \item \fnref{submap}\code{(keyspec)\{ ... more key bindings ... \}}.
\end{itemize}
Mouse actions:
\begin{itemize}
    \item \fnref{mclick}\code{(buttonspec, func, [, area])},
    \item \fnref{mdblclick}\code{(buttonspec, func, [, area])}, 
    \item \fnref{mpress}\code{(buttonspec, func, [, area])} and
    \item \fnref{mdrag}\code{(buttonspec, func, [, area])}.
\end{itemize}

The actions that most of these functions correspond to should be clear
and as explained in the reference, \fnref{kpress_waitrel} is simply
\fnref{kpress} with a flag set instructing Ioncore wait for all
modifiers to be released before processing any further actions.
This is to stop one from accidentally calling e.g.
\fnref{WRegion.close} multiple times in a row. The \fnref{submap}
function actually returns another function that expects a table
as an argument if called with only one argument. Alternatively you
could use \code{submap(keyspec, \{ ... \})}.

The parameters \var{keyspec} and \var{buttonspec} are explained below
in detail. The parameter \var{func} is the handler for the binding.
It is a reference to a function that should expect as parameter(s)
objects of the type defined in the above tables depending on which 
binding  setup function this binding definition is being passed to.

The optional string parameter \var{area} may be used to specify a more
precise location where the mouse action should occur for the binding to
take place.
Currently only \fnref{ionframe_bindings} and \fnref{floatframe_bindings}
support any meaningfull values for this parameter. The allowed values
are in this case
\code{"border"}, \code{"tab"}, \code{"empty_tab"}, \code{"client"} and
\code{nil} (for the whole frame).

\subsection{An example}

For example, to just bind the key \key{Mod1+1} to switch to the first
workspace and \key{Mod1+n} to the next workspace, you would make the
following call
\begin{verbatim}
global_bindings{
    kpress("Mod1+n", WScreen.switch_nth),
    kpress("Mod1+1", function (scr) scr:switch_nth(1) end),
}
\end{verbatim}

Recall that \code{global_bindings\{...\}} is syntactical sugar for
the more cumbersome
\code{global_bindings(\{...\})}.

The first definition works, because \code{WScreen.switch_next}
(inherited \fnref{WMPlex.switch_next}) is a function that takes a 
\type{WScreen} as its sole parameter. On the other hand,
\code{WScreen.switch_nth} (again inherited \fnref{WMPlex.switch_nth}) 
expects two parameters: the screen and the $n$ giving the number of
the workspace to switch to so it could not be directly passed to
\code{kpress}. Therefore we had to write our ''anonymous'' wrapper
function around it tha takes a single parameter and calls
\code{switch_nth} properly. (Recall that \code{scr:switch_nth(1)}
is syntactic sugar for \code{WScreen.switch_nth(scr, 1)}.
Alternatively we could have defined
\code{function switch_1(scr) scr:switch_nth(1) end} before the call
to \code{global_bindings} and passed \code{switch_1} to \code{kpress}
instead of the anonymous function.)

\subsection{Key and button specifications}

As seen above, the functions that create key binding specifications require
a \var{keyspec} argument. This argument should be a string containing the
name of a key as listed in the X header file \file{keysymdef.h}%
\footnote{This file can usually be found in the directory
\file{/usr/X11R6/include/X11/}.} without the \code{XK_} prefix.
\index{keysymdef.h@\file{keysymdef.h}}
Most of the key names are quite intuitive while some are not. For example,
the \key{Enter} key on the main part of the keyboard has the less common
name \key{Return} while the one one the numpad is called \key{KP\_Enter}.

The \var{keyspec} string may optionally have multiple ''modifier'' names
followed by a plus sign (\code{+}) as a prefix. X defines the following
modifiers:
\begin{quotation}
\key{Shift}, \key{Control}, \key{Mod1} to \key{Mod5},
\key{AnyModifier} and \key{Lock}.
\index{Shift@\key{Shift}}
\index{Control@\key{Control}}
\index{Mod-n@\key{Mod-n}}
\index{AnyModifier@\key{AnyModifier}}
\index{Lock@\key{Lock}}
\end{quotation}

X allows binding all of these modifiers to almost any key and while this
list of modifiers does not explicitly list keys such as 
\key{Alt}\index{Alt@\key{Alt}} that are common on modern keyboards, such
keys are bound to one of the \key{Mod-n}. On systems running XFree86
\key{Alt} is usually \key{Mod1}. On Suns \key{Mod1} is the diamond key
and \key{Alt} something else. One of the ''flying window'' keys on so
called Windows-keyboards is probably mapped to \key{Mod3} if you have
such a key. Use the program \file{xmodmap}\index{xmodmap@\file{xmodmap}}
to find out what exactly is bound where. \key{AnyModifier} is usually
used in submaps to indicate that it doesn't matter which modifier keys
are pressed, if any.

Ion ignores the \key{Lock} modifier and any \key{Mod-n} modifiers bound
to 
\key{NumLock}\index{NumLock@\key{NumLock}} or
\key{ScrollLock}\index{ScrollLock@\key{ScrollLock}}
by default because such\footnote{Completely useless keys that should be
  gotten rid of in the author's opinion.} locking keys may otherwise
cause confusion.

Button specifications are similar to key definitions but now
instead of specifying modifiers and a key, you specify modifiers
and one of the button names \key{Button1} to
\key{Button5}\index{Button-n@\key{Button-n}}.


\subsection{Another example}

TODO

\subsection{A further note on the default binding configuration}

The variable \code{DEFAULT_MOD} in the above listing defaults to
\code{"Mod1+"} and is set in \file{ion.lua}. Changing this
variable allows to easily change the the modifier used by all bindings
in the default configuration that use modifiers. Quite a few people
prefer to use the Windows keys as modifiers because many applications
already use \key{Alt}. Nevertheless, \key{Mod1} is the default as a
key bound to it is available virtually everywhere.

\subsection{Client window bindings}

As client windows do not have function to set their bindings, it is
necessary to call client window functions by specifying the bindings
somewhere else. In the stock configuration file setup this is done
in \fnref{mplex_bindings} by functions that look up the object currently
displayed by the \type{WMPlex} (\fnref{WMPlex.current}). We then check that
it is of type \type{WClientWin} to suppress warning and then call the
wanted function with the verified client window as argument. 

To make it easier to write such bindings, the function 
\fnref{make_mplex_clientwin_fn} is used to construct this wrapper
function. The following two binding definitions are essentially
equivalent:

\begin{verbatim}
mplex_bindings {
    kpress("Mod1+Return", function(mplex, r)
                              if not r or r==mplex
                                  r=mplex:current()
                              end
                              if obj_is(r, "WClientWin") then
                                  r:toggle_fullscreen()
                              end
                          end),
    kpress("Mod1+Enter",
           make_mplex_clientwin_fn(WClientWin.toggle_fullscreen)),
}
\end{verbatim}

%In the stock configuration files, a similar function
%\fnref{make_current_or_self_fn} is also used to create a wrapper for
%\fnref{region_close} that attempts to close the current object managed
%by the \type{WMPlex} if it exists and otherwise the multiplexer itself.


\section{Winprops}
\label{sec:winprops}

\subsection{Classes, roles and instances}

The so-called ''winprops''\index{Winprops} can be used to change how
specific windows are handled and to set up some kludges to deal with
badly behaving applications. They are defined by calling the function
\code{winprop} with a table containing the properties to set and the
necessary information to identify a window. This identification
information is more specifically the
\var{class}\index{class@\var{class}!winprop},
\var{role}\index{role@\var{role}!winprop} and
\var{instance}\index{instance@\var{instance}!winprop}
\var{name}
of the window. The \var{name} field is a Lua-style regular expression
matched against the window's title and the rest are strings that must
exactly much the corresponding window information. It is not necessary
to specify all of these fields.

Ion looks for a matching winprop in the order listed by the following
table. An 'E' indicates that the field must be set in the winprop
and it must match the window's corresponding propertyexactly or, in
case of \var{name}, the regular expression must match the window
title. An asterisk '*' indicates that a winprop where the field is
not specified (or is itself an asterisk in case of the first three
fields) is tried.

\begin{center}
\begin{tabular}{llll}
\hline
\var{class} & \var{role} & \var{instance} & \var{name} \\
\hline
  E	       & E          & E              & E \\
  E	       & E          & E              & * \\
  E	       & E          & *              & E \\
  E	       & E          & *              & * \\
  E	       & *          & E              & E \\
  E	       & *          & E              & * \\
  E	       & *          & *              & E \\
  \vdots       & \vdots     & \vdots         & etc. \\
\end{tabular}
\end{center}

If there are multiple winprops with other identification information 
the same but different \var{name}, the longest match is chosen.

To get this identification information for a particular window, you
may use the command \command{xprop WM_CLASS} and click on that
particular window.\footnote{This does not work for transients in
WIonFrames.} The class is the latter of the strings while
the instance is the former. To get the role -- few windows have
this property -- use the command \command{xprop WM_ROLE}.

\subsection{Supported winprops}

Ion currently knows the following winprops:

\index{switchto@\var{switchto}!winprop}
\index{transient-mode@\var{transient_mode}!winprop}
\index{target@\var{target}!winprop}
\index{transparent@\var{transparent}!winprop}
\index{acrobatic@\var{acrobatic}!winprop}
\index{max-size@\var{max_size}!winprop}
\index{aspect@\var{aspect}!winprop}
\index{ignore-resizeinc@\var{ignore_resizeinc}}

\begin{tabularx}{\textwidth}{llX}
    \hline
    Property & Type & Description\\\hline
    \var{switchto} &
    	boolean &
    	Should the window be switched to when it is created. \\
    \var{transient_mode} &
  	string &
    	"normal": No change in behaviour. "current": The window
	should be thought of as a transient for the current active
	client window (if any) even if it is not marked as a
	transient by the application. "off": The window should be
	handled as a normal window even if it is marked as a
	transient by the application. \\
    \var{target} &
    	string &
    	The name of an object (workspace, frame) that should manage 
	windows of this type. \\
    \var{transparent} &
    	boolean &
    	Should frames be made transparent when this window is selected? \\
    \var{acrobatic} &
    	boolean &
    	Set this to \code{true} for Acrobat Reader. It has an annoying
	habit of trying to manage its dialogs instead of setting them as
	transients and letting the window manager do its job, causing
	Ion and acrobat go a window-switching loop when a dialog is
	opened. \\
    \var{max_size} &
    	table &
        The table should contain the entries \var{w} and \var{h} that
	override application-supplied maximum size hint. \\
    \var{aspect} &
    	table &
        The table should contain the entries \var{w} and \var{h} that
	override application-supplied aspect ratio hint. \\
    \var{ignore_resizeinc} &
    	boolean &
    	Should application supplied size increments be ignored? \\
    \var{fullscreen} &
    	boolean &
    	Should the window be initially in full screen mode? \\
    \var{ignore_cfgrq} &
    	boolean &
    	Should configure requests on the window be ignored?
	Only has effect on windows on floatws:s. \\
    \var{transients_at_top} &
    	boolean &
    	When transients are managed by the client window itself (as it
	is the case on tiled workspaces), should the transients be
	placed at the top of the window instead of bottom? \\
\end{tabularx}

\subsection{Examples}

Acrobat Reader's manners aren't exactly good:
\begin{verbatim}
winprop{
    class = "AcroRead",
    instance = "documentShell",
    acrobatic = true,
}
\end{verbatim}

Place xterm started with '\code{-name sysmon}' and running a system
monitoring program in a specific frame:
\begin{verbatim}
winprop{
    class = "XTerm",
    instance = "sysmon",
    target = "sysmonframe",
}
\end{verbatim}


\section{The query library}
\index{querylib@\var{querylib}}

The query module does not implement any queries in itself, but provides
the function \fnref{query_query} to execute arbitrary queries. Some
standard queries implemented with this interface are available in 
'querylib' (\file{SHAREDIR/querylib.lua}). Most of these queries support
tab-completion and can be directly passed to the binding setup functions
for classes based on \type{WMPlex}.  These setup functions are at the moment:
\begin{itemize}
\item \fnref{global_bindings}
\item \fnref{mplex_bindings}, 
\item \fnref{genframe_bindings},
\item \fnref{ionframe_bindings} and
\item \fnref{floatframe_bindings}.
\end{itemize}

The default configuration puts most queries in \fnref{genframe_bindings}
while the exit and restart queries are in \fnref{global_bindings}.
For a listing of the functions, see section \ref{sec:querylibref} in the 
Function referece. 

Querylib also provides functions to generate more queries; for details see
the script.

\section{Menus}
\label{sec:menus}

\subsection{Defining menus}

\index{menus}
\index{defmenu@\code{defmenu}}
\index{menuentry@\code{menuentry}}
\index{submenu@\code{submenu}}
In the stock configuration file setup, menus are configured in the file
\file{ion-menus.lua} as previously mentioned. An example of a 
definition of a rather simple menu with a submenu is:
\begin{verbatim}
include("menulib")

defmenu("exitmenu", {
    menuentry("Restart", restart_wm),
    menuentry("Exit", exit_wm),
})

defmenu("mainmenu", {
    menuentry("Lock screen", make_exec_fn("xlock")),
    menuentry("Help", querylib.query_man),
    submenu("Exit", "exitmenu"),
})
\end{verbatim}

The \file{menulib} library must be loaded for some of the functions 
discussed here to be available.

The \fnref{defmenu} function is used to define a named menu that can later
be accessed with this name. The \fnref{menuentry} function is used to
create an entry in the menu with a title and an entry handler function to
be called when the menu entry is activated. If the functions discussed in
subsection \ref{sec:menudisp} are used to display the menu from a binding, 
the  parameters that are passed to the function are those that the binding
handler was passed. It is as if the function was called from that binding. 

The \fnref{submenu} function is used to insert a submenu at that point in
the menu. (One could as well just pass a table with the menu
entries, but it is not encouraged.)

\subsection{Special menus}

The \file{menulib} library predefines the following special menus.
These can be used as the menus defined as above.

\begin{tabularx}{\linewidth}{lX}
    \hline
    Menu name & Description \\
    \hline
    \code{windowlist} & 
    List of all client windows. Activing an entry jumps to that window. \\
    \code{workspacelist} & 
    List of all workspaces. Activating an entry jumps to that workspaces. \\
    \code{stylemenu} &
    List of available \file{look-*.lua} style files. Activating an entry
    loads that style and ask to save the selection. \\
\end{tabularx}


\subsection{Displaying menus}
\label{sec:menudisp}

\index{make_menu_fn@\code{make_menu_fn}}
\index{make_bigmenu_fn@\code{make_bigmenu_fn}}
\index{make_pmenu_fn@\code{make_pmenu_fn}}
Menus defined with the routines described in the previous subsection
should be bound to key and pointer actions by creating a bindable
function with one of the following routines: \fnref{make_menu_fn},
\fnref{make_bigmenu_fn} or \fnref{make_pmenu_fn}. The first two
create functions to display in-frame (or in-mplex more generally)
menus that appear on the bottom-left corner of the \type{WMPlex}
where the bound action occured. The difference between the two is
the different drawing engine style used. The last function creates
a pop-up menu display function and can only be bound to mouse press
actions.

An example of a binding to display a menu is:
\begin{verbatim}
global_bindings{
    kpress("F12", make_bigmenu_fn("mainmenu")),
}
\end{verbatim}

The low-level functions \fnref{menu_menu} and \fnref{menu_pmenu} can
also be used to display menus with different kinds of handlers and so
on, but most users should not need to be concerned with these.
If you use these functions note that they do not call the menu entry
handlers but pass the entry to a specified handler the responsibility
of which it is to decide what to do.


\section{Some common configuration tasks}

\subsection{Binding a key to execute a program}

Because the \fnref{exec} function immediately executes the argument string,
we must wrap this function in the bindings in the following way:
\begin{verbatim}
    kpress("SomeKey", function() exec("program --param") end)
\end{verbatim}
The \fnref{make_exec_fn} function can be used as a convenience; the
above is equivalent to
\begin{verbatim}
    kpress("SomeKey", make_exec_fn("program --param"))
\end{verbatim}


\chapter{Function reference}
\label{sec:exports}

The following subsections list out the functions exported to Lua scripts
by Ioncore and the ionws, floatws and query modules.

\section{Functions exported by Ioncore}
\label{sec:ioncoreref}

\input{ioncore-exports.tex}

\section{Functions exported by the ionws module}
\label{sec:ionwsref}

\input{ionws-exports.tex}

\section{Functions exported by the floatws module}
\label{sec:floatwsref}

\input{floatws-exports.tex}

\section{Functions exported by the query module}
\label{sec:queryref}

\input{query-exports.tex}

\section{Functions defined by \file{ioncorelib.lua} and files included
therefrom}
\label{sec:ioncorelibref}

\input{ioncorelib-fns.tex}

\section{Functions defined by \file{ioncore-mplexfns.lua}}
\label{sec:mplexfnsref}

\input{ioncore-mplexfns.tex}

\section{Functions defined by \file{querylib.lua}}
\label{sec:querylibref}

\input{querylib-fns.tex}

\appendix

\xchapter{GNU Free Documentation License}

 \begin{center}

       Version 1.2, November 2002


 Copyright \copyright{} 2000,2001,2002  Free Software Foundation, Inc.
 
 \bigskip
 
     51 Franklin St, Fifth Floor, Boston, MA  02110-1301  USA
  
 \bigskip
 
 Everyone is permitted to copy and distribute verbatim copies
 of this license document, but changing it is not allowed.
\end{center}


\begin{center}
{\bf\large Preamble}
\end{center}

The purpose of this License is to make a manual, textbook, or other
functional and useful document ``free'' in the sense of freedom: to
assure everyone the effective freedom to copy and redistribute it,
with or without modifying it, either commercially or noncommercially.
Secondarily, this License preserves for the author and publisher a way
to get credit for their work, while not being considered responsible
for modifications made by others.

This License is a kind of ``copyleft'', which means that derivative
works of the document must themselves be free in the same sense.  It
complements the GNU General Public License, which is a copyleft
license designed for free software.

We have designed this License in order to use it for manuals for free
software, because free software needs free documentation: a free
program should come with manuals providing the same freedoms that the
software does.  But this License is not limited to software manuals;
it can be used for any textual work, regardless of subject matter or
whether it is published as a printed book.  We recommend this License
principally for works whose purpose is instruction or reference.


\begin{center}
{\large\bf 1. APPLICABILITY AND DEFINITIONS\par}
\phantomsection
%\addcontentsline{toc}{section}{1. APPLICABILITY AND DEFINITIONS}
\end{center}

This License applies to any manual or other work, in any medium, that
contains a notice placed by the copyright holder saying it can be
distributed under the terms of this License.  Such a notice grants a
world-wide, royalty-free license, unlimited in duration, to use that
work under the conditions stated herein.  The ``\textbf{Document}'', below,
refers to any such manual or work.  Any member of the public is a
licensee, and is addressed as ``\textbf{you}''.  You accept the license if you
copy, modify or distribute the work in a way requiring permission
under copyright law.

A ``\textbf{Modified Version}'' of the Document means any work containing the
Document or a portion of it, either copied verbatim, or with
modifications and/or translated into another language.

A ``\textbf{Secondary Section}'' is a named appendix or a front-matter section of
the Document that deals exclusively with the relationship of the
publishers or authors of the Document to the Document's overall subject
(or to related matters) and contains nothing that could fall directly
within that overall subject.  (Thus, if the Document is in part a
textbook of mathematics, a Secondary Section may not explain any
mathematics.)  The relationship could be a matter of historical
connection with the subject or with related matters, or of legal,
commercial, philosophical, ethical or political position regarding
them.

The ``\textbf{Invariant Sections}'' are certain Secondary Sections whose titles
are designated, as being those of Invariant Sections, in the notice
that says that the Document is released under this License.  If a
section does not fit the above definition of Secondary then it is not
allowed to be designated as Invariant.  The Document may contain zero
Invariant Sections.  If the Document does not identify any Invariant
Sections then there are none.

The ``\textbf{Cover Texts}'' are certain short passages of text that are listed,
as Front-Cover Texts or Back-Cover Texts, in the notice that says that
the Document is released under this License.  A Front-Cover Text may
be at most 5 words, and a Back-Cover Text may be at most 25 words.

A ``\textbf{Transparent}'' copy of the Document means a machine-readable copy,
represented in a format whose specification is available to the
general public, that is suitable for revising the document
straightforwardly with generic text editors or (for images composed of
pixels) generic paint programs or (for drawings) some widely available
drawing editor, and that is suitable for input to text formatters or
for automatic translation to a variety of formats suitable for input
to text formatters.  A copy made in an otherwise Transparent file
format whose markup, or absence of markup, has been arranged to thwart
or discourage subsequent modification by readers is not Transparent.
An image format is not Transparent if used for any substantial amount
of text.  A copy that is not ``Transparent'' is called ``\textbf{Opaque}''.

Examples of suitable formats for Transparent copies include plain
ASCII without markup, Texinfo input format, LaTeX input format, SGML
or XML using a publicly available DTD, and standard-conforming simple
HTML, PostScript or PDF designed for human modification.  Examples of
transparent image formats include PNG, XCF and JPG.  Opaque formats
include proprietary formats that can be read and edited only by
proprietary word processors, SGML or XML for which the DTD and/or
processing tools are not generally available, and the
machine-generated HTML, PostScript or PDF produced by some word
processors for output purposes only.

The ``\textbf{Title Page}'' means, for a printed book, the title page itself,
plus such following pages as are needed to hold, legibly, the material
this License requires to appear in the title page.  For works in
formats which do not have any title page as such, ``Title Page'' means
the text near the most prominent appearance of the work's title,
preceding the beginning of the body of the text.

A section ``\textbf{Entitled XYZ}'' means a named subunit of the Document whose
title either is precisely XYZ or contains XYZ in parentheses following
text that translates XYZ in another language.  (Here XYZ stands for a
specific section name mentioned below, such as ``\textbf{Acknowledgements}'',
``\textbf{Dedications}'', ``\textbf{Endorsements}'', or ``\textbf{History}''.)  
To ``\textbf{Preserve the Title}''
of such a section when you modify the Document means that it remains a
section ``Entitled XYZ'' according to this definition.

The Document may include Warranty Disclaimers next to the notice which
states that this License applies to the Document.  These Warranty
Disclaimers are considered to be included by reference in this
License, but only as regards disclaiming warranties: any other
implication that these Warranty Disclaimers may have is void and has
no effect on the meaning of this License.


\begin{center}
{\large\bf 2. VERBATIM COPYING\par}
\phantomsection
%\addcontentsline{toc}{section}{2. VERBATIM COPYING}
\end{center}

You may copy and distribute the Document in any medium, either
commercially or noncommercially, provided that this License, the
copyright notices, and the license notice saying this License applies
to the Document are reproduced in all copies, and that you add no other
conditions whatsoever to those of this License.  You may not use
technical measures to obstruct or control the reading or further
copying of the copies you make or distribute.  However, you may accept
compensation in exchange for copies.  If you distribute a large enough
number of copies you must also follow the conditions in section~3.

You may also lend copies, under the same conditions stated above, and
you may publicly display copies.


\begin{center}
{\large\bf 3. COPYING IN QUANTITY\par}
\phantomsection
%\addcontentsline{toc}{section}{3. COPYING IN QUANTITY}
\end{center}


If you publish printed copies (or copies in media that commonly have
printed covers) of the Document, numbering more than 100, and the
Document's license notice requires Cover Texts, you must enclose the
copies in covers that carry, clearly and legibly, all these Cover
Texts: Front-Cover Texts on the front cover, and Back-Cover Texts on
the back cover.  Both covers must also clearly and legibly identify
you as the publisher of these copies.  The front cover must present
the full title with all words of the title equally prominent and
visible.  You may add other material on the covers in addition.
Copying with changes limited to the covers, as long as they preserve
the title of the Document and satisfy these conditions, can be treated
as verbatim copying in other respects.

If the required texts for either cover are too voluminous to fit
legibly, you should put the first ones listed (as many as fit
reasonably) on the actual cover, and continue the rest onto adjacent
pages.

If you publish or distribute Opaque copies of the Document numbering
more than 100, you must either include a machine-readable Transparent
copy along with each Opaque copy, or state in or with each Opaque copy
a computer-network location from which the general network-using
public has access to download using public-standard network protocols
a complete Transparent copy of the Document, free of added material.
If you use the latter option, you must take reasonably prudent steps,
when you begin distribution of Opaque copies in quantity, to ensure
that this Transparent copy will remain thus accessible at the stated
location until at least one year after the last time you distribute an
Opaque copy (directly or through your agents or retailers) of that
edition to the public.

It is requested, but not required, that you contact the authors of the
Document well before redistributing any large number of copies, to give
them a chance to provide you with an updated version of the Document.


\begin{center}
{\large\bf 4. MODIFICATIONS\par}
\phantomsection
%\addcontentsline{toc}{section}{4. MODIFICATIONS}
\end{center}

You may copy and distribute a Modified Version of the Document under
the conditions of sections 2 and 3 above, provided that you release
the Modified Version under precisely this License, with the Modified
Version filling the role of the Document, thus licensing distribution
and modification of the Modified Version to whoever possesses a copy
of it.  In addition, you must do these things in the Modified Version:

\begin{itemize}
\item[A.] 
   Use in the Title Page (and on the covers, if any) a title distinct
   from that of the Document, and from those of previous versions
   (which should, if there were any, be listed in the History section
   of the Document).  You may use the same title as a previous version
   if the original publisher of that version gives permission.
   
\item[B.]
   List on the Title Page, as authors, one or more persons or entities
   responsible for authorship of the modifications in the Modified
   Version, together with at least five of the principal authors of the
   Document (all of its principal authors, if it has fewer than five),
   unless they release you from this requirement.
   
\item[C.]
   State on the Title page the name of the publisher of the
   Modified Version, as the publisher.
   
\item[D.]
   Preserve all the copyright notices of the Document.
   
\item[E.]
   Add an appropriate copyright notice for your modifications
   adjacent to the other copyright notices.
   
\item[F.]
   Include, immediately after the copyright notices, a license notice
   giving the public permission to use the Modified Version under the
   terms of this License, in the form shown in the Addendum below.
   
\item[G.]
   Preserve in that license notice the full lists of Invariant Sections
   and required Cover Texts given in the Document's license notice.
   
\item[H.]
   Include an unaltered copy of this License.
   
\item[I.]
   Preserve the section Entitled ``History'', Preserve its Title, and add
   to it an item stating at least the title, year, new authors, and
   publisher of the Modified Version as given on the Title Page.  If
   there is no section Entitled ``History'' in the Document, create one
   stating the title, year, authors, and publisher of the Document as
   given on its Title Page, then add an item describing the Modified
   Version as stated in the previous sentence.
   
\item[J.]
   Preserve the network location, if any, given in the Document for
   public access to a Transparent copy of the Document, and likewise
   the network locations given in the Document for previous versions
   it was based on.  These may be placed in the ``History'' section.
   You may omit a network location for a work that was published at
   least four years before the Document itself, or if the original
   publisher of the version it refers to gives permission.
   
\item[K.]
   For any section Entitled ``Acknowledgements'' or ``Dedications'',
   Preserve the Title of the section, and preserve in the section all
   the substance and tone of each of the contributor acknowledgements
   and/or dedications given therein.
   
\item[L.]
   Preserve all the Invariant Sections of the Document,
   unaltered in their text and in their titles.  Section numbers
   or the equivalent are not considered part of the section titles.
   
\item[M.]
   Delete any section Entitled ``Endorsements''.  Such a section
   may not be included in the Modified Version.
   
\item[N.]
   Do not retitle any existing section to be Entitled ``Endorsements''
   or to conflict in title with any Invariant Section.
   
\item[O.]
   Preserve any Warranty Disclaimers.
\end{itemize}

If the Modified Version includes new front-matter sections or
appendices that qualify as Secondary Sections and contain no material
copied from the Document, you may at your option designate some or all
of these sections as invariant.  To do this, add their titles to the
list of Invariant Sections in the Modified Version's license notice.
These titles must be distinct from any other section titles.

You may add a section Entitled ``Endorsements'', provided it contains
nothing but endorsements of your Modified Version by various
parties--for example, statements of peer review or that the text has
been approved by an organization as the authoritative definition of a
standard.

You may add a passage of up to five words as a Front-Cover Text, and a
passage of up to 25 words as a Back-Cover Text, to the end of the list
of Cover Texts in the Modified Version.  Only one passage of
Front-Cover Text and one of Back-Cover Text may be added by (or
through arrangements made by) any one entity.  If the Document already
includes a cover text for the same cover, previously added by you or
by arrangement made by the same entity you are acting on behalf of,
you may not add another; but you may replace the old one, on explicit
permission from the previous publisher that added the old one.

The author(s) and publisher(s) of the Document do not by this License
give permission to use their names for publicity for or to assert or
imply endorsement of any Modified Version.


\begin{center}
{\large\bf 5. COMBINING DOCUMENTS\par}
\phantomsection
%\addcontentsline{toc}{section}{5. COMBINING DOCUMENTS}
\end{center}


You may combine the Document with other documents released under this
License, under the terms defined in section~4 above for modified
versions, provided that you include in the combination all of the
Invariant Sections of all of the original documents, unmodified, and
list them all as Invariant Sections of your combined work in its
license notice, and that you preserve all their Warranty Disclaimers.

The combined work need only contain one copy of this License, and
multiple identical Invariant Sections may be replaced with a single
copy.  If there are multiple Invariant Sections with the same name but
different contents, make the title of each such section unique by
adding at the end of it, in parentheses, the name of the original
author or publisher of that section if known, or else a unique number.
Make the same adjustment to the section titles in the list of
Invariant Sections in the license notice of the combined work.

In the combination, you must combine any sections Entitled ``History''
in the various original documents, forming one section Entitled
``History''; likewise combine any sections Entitled ``Acknowledgements'',
and any sections Entitled ``Dedications''.  You must delete all sections
Entitled ``Endorsements''.

\begin{center}
{\large\bf 6. COLLECTIONS OF DOCUMENTS\par}
\phantomsection
%\addcontentsline{toc}{section}{6. COLLECTIONS OF DOCUMENTS}
\end{center}

You may make a collection consisting of the Document and other documents
released under this License, and replace the individual copies of this
License in the various documents with a single copy that is included in
the collection, provided that you follow the rules of this License for
verbatim copying of each of the documents in all other respects.

You may extract a single document from such a collection, and distribute
it individually under this License, provided you insert a copy of this
License into the extracted document, and follow this License in all
other respects regarding verbatim copying of that document.


\begin{center}
{\large\bf 7. AGGREGATION WITH INDEPENDENT WORKS\par}
\phantomsection
%\addcontentsline{toc}{section}{7. AGGREGATION WITH INDEPENDENT WORKS}
\end{center}


A compilation of the Document or its derivatives with other separate
and independent documents or works, in or on a volume of a storage or
distribution medium, is called an ``aggregate'' if the copyright
resulting from the compilation is not used to limit the legal rights
of the compilation's users beyond what the individual works permit.
When the Document is included in an aggregate, this License does not
apply to the other works in the aggregate which are not themselves
derivative works of the Document.

If the Cover Text requirement of section~3 is applicable to these
copies of the Document, then if the Document is less than one half of
the entire aggregate, the Document's Cover Texts may be placed on
covers that bracket the Document within the aggregate, or the
electronic equivalent of covers if the Document is in electronic form.
Otherwise they must appear on printed covers that bracket the whole
aggregate.


\begin{center}
{\large\bf 8. TRANSLATION\par}
\phantomsection
%\addcontentsline{toc}{section}{8. TRANSLATION}
\end{center}


Translation is considered a kind of modification, so you may
distribute translations of the Document under the terms of section~4.
Replacing Invariant Sections with translations requires special
permission from their copyright holders, but you may include
translations of some or all Invariant Sections in addition to the
original versions of these Invariant Sections.  You may include a
translation of this License, and all the license notices in the
Document, and any Warranty Disclaimers, provided that you also include
the original English version of this License and the original versions
of those notices and disclaimers.  In case of a disagreement between
the translation and the original version of this License or a notice
or disclaimer, the original version will prevail.

If a section in the Document is Entitled ``Acknowledgements'',
``Dedications'', or ``History'', the requirement (section~4) to Preserve
its Title (section~1) will typically require changing the actual
title.


\begin{center}
{\large\bf 9. TERMINATION\par}
\phantomsection
%\addcontentsline{toc}{section}{9. TERMINATION}
\end{center}


You may not copy, modify, sublicense, or distribute the Document except
as expressly provided for under this License.  Any other attempt to
copy, modify, sublicense or distribute the Document is void, and will
automatically terminate your rights under this License.  However,
parties who have received copies, or rights, from you under this
License will not have their licenses terminated so long as such
parties remain in full compliance.


\begin{center}
{\large\bf 10. FUTURE REVISIONS OF THIS LICENSE\par}
\phantomsection
%\addcontentsline{toc}{section}{10. FUTURE REVISIONS OF THIS LICENSE}
\end{center}


The Free Software Foundation may publish new, revised versions
of the GNU Free Documentation License from time to time.  Such new
versions will be similar in spirit to the present version, but may
differ in detail to address new problems or concerns.  See
\url{http://www.gnu.org/copyleft/}.

Each version of the License is given a distinguishing version number.
If the Document specifies that a particular numbered version of this
License ``or any later version'' applies to it, you have the option of
following the terms and conditions either of that specified version or
of any later version that has been published (not as a draft) by the
Free Software Foundation.  If the Document does not specify a version
number of this License, you may choose any version ever published (not
as a draft) by the Free Software Foundation.


\begin{center}
{\large\bf ADDENDUM: How to use this License for your documents\par}
\phantomsection
%\addcontentsline{toc}{section}{ADDENDUM: How to use this License for your documents}
\end{center}

To use this License in a document you have written, include a copy of
the License in the document and put the following copyright and
license notices just after the title page:

\bigskip
\begin{quote}
    Copyright \copyright{}  YEAR  YOUR NAME.
    Permission is granted to copy, distribute and/or modify this document
    under the terms of the GNU Free Documentation License, Version 1.2
    or any later version published by the Free Software Foundation;
    with no Invariant Sections, no Front-Cover Texts, and no Back-Cover Texts.
    A copy of the license is included in the section entitled ``GNU
    Free Documentation License''.
\end{quote}
\bigskip
    
If you have Invariant Sections, Front-Cover Texts and Back-Cover Texts,
replace the ``with \dots\ Texts.'' line with this:

\bigskip
\begin{quote}
    with the Invariant Sections being LIST THEIR TITLES, with the
    Front-Cover Texts being LIST, and with the Back-Cover Texts being LIST.
\end{quote}
\bigskip
    
If you have Invariant Sections without Cover Texts, or some other
combination of the three, merge those two alternatives to suit the
situation.

If your document contains nontrivial examples of program code, we
recommend releasing these examples in parallel under your choice of
free software license, such as the GNU General Public License,
to permit their use in free software.



\printindex

\end{document}
